\documentclass[toc=chapterentrywithdots,a4paper,fontsize=12pt,listof=totoc,bibliography=totoc]{scrbook}
\pagestyle{plain} % no running headers, but page number in footer as required by FHBgld
%--Packages added by Dominik Thiede
\usepackage[table,xcdraw]{xcolor} % Use colours in table
\usepackage{pifont} %Use Latex Symbols like check mark 
\newcommand{\cmark}{\ding{51}}
\newcommand{\xmark}{\ding{55}}
\newcommand{\omark}{\ding{72}}
\usepackage{listings} % line break for programm code
\usepackage{xcolor} % highlighting text
\usepackage{glossaries}
% -- Packages needed in the class
\usepackage{scrhack}
%\usepackage[a-1b]{pdfx} % Required for PDF-A compliance 
\usepackage[left=2.5cm,right=2.5cm,top=2.5cm,bottom=2cm,bindingoffset=1cm]{geometry} % required for exact FHBgld page margins
%if changed to \usepackage[ngerman]{babel} "Table of contents" changes to "Inhaltsverzeichnis", "abstract" becomes "Zusammenfassung" and "references" is translated to "Literatur".
\usepackage[ngerman, british]{babel} 
\usepackage{graphicx} % Required for including pictures
%\usepackage[pdfa]{hyperref} % Required for PDF-A compliance 
\usepackage[hidelinks,pdfpagelayout=TwoPageRight]{hyperref} % FHBgld - Hide links: border and color
%\usepackage[hidelinks]{hyperref}
\usepackage[T1]{fontenc} % Required for accented characters
%\usepackage[utf8]{inputenc} % legacy, not needed
\usepackage{changepage} % detects odd/even pages
\usepackage[final]{microtype}
\usepackage{xurl}
\usepackage{mathpazo} % FHBgld required "Book Antiqua" font is a clone of Palatino
% \usepackage[natbib=true, backend=biber, style=apa, sorting=nty]{biblatex} % FHBgld APA bibliography style
%\addbibresource{bibliography.bib} % Bibilography resource file % this goes into the main.tex, otherwise some editors wont find it
\setcounter{secnumdepth}{4} % FHBgld Numbering depth
%\setcounter{tocdepth}{3} % FHBgld TOC depth
\usepackage{csquotes} % Context sensitive quotation facilities, needed for babel and Biblatex-APA to work
\usepackage{booktabs} % Publication quality tables in LaTeX
\usepackage{tabularx} % Tabulars with adjustable-width columns.
\usepackage{multirow} % Create tabular cells spanning multiple rows
\usepackage{algorithm2e} % Package to show algorithms
\usepackage{listings} % Package to format code
\usepackage{xcolor} % Driver-independent color extensions for LaTeX
%\usepackage[super]{nth} % superscript
\usepackage{fmtcount} % another superscript ;-)
\usepackage{verbatim} % provides \begin{verbatim} formatting
\usepackage{array} % additional functionality for the tabular environment

\usepackage[record=only,acronym,nonumberlist,nopostdot=false,toc=true,style=super]{glossaries-extra} % provides extra glossaries features plus bib2gls, use convertgls2bib  defns.tex defns.bib to convert
\usepackage[justification=RaggedRight,singlelinecheck=false,font=footnotesize]{caption} % Needed to format captions to align left
%\RequirePackage{tocloft} % allows to change directories

% \usepackage{showframe}  % Draw a page-layout diagram on pages, RMF

\usepackage{lipsum} % lipsum :-) , RMF 

%  -- colors
\definecolor{codegreen}{rgb}{0,0.6,0}
\definecolor{codegray}{rgb}{0.5,0.5,0.5}
\definecolor{codepurple}{rgb}{0.58,0,0.82}
\definecolor{backcolour}{rgb}{0.95,0.95,0.92}

%  -- FHBgld fonts
% regular fontsize=12pt set in scrbook and mathpazo for Palatino-Roman above
\setkomafont{disposition}{\bfseries} % bold+fat for all sectioning command titles
\addtokomafont{title}{\fontsize{24}{24}\selectfont} % FHBgld Title
\addtokomafont{subtitle}{\fontsize{14}{16}\selectfont} % FHBgld Subtitle
%\addtokomafont{part}{\fontsize{16}{16}\selectfont} % FHBgld Level 1
\addtokomafont{chapter}{\fontsize{16}{16}\selectfont} % FHBgld Level 1
\addtokomafont{section}{\fontsize{14}{16}\selectfont} % FHBgld Level 2
\addtokomafont{subsection}{\fontsize{14}{16}\selectfont} % FHBgld Level 3
\addtokomafont{subsubsection}{\fontsize{13}{16}\selectfont} % FHBgld Level 4

% -- Default table fontsize according to FHBgld requirements
\let\oldtabular\tabular 
\renewcommand{\tabular}{\fontsize{10pt}{14pt}\selectfont\oldtabular}
% FHBgld wants the first line to have 11pt, unfortunately the array package lets you configure columns only. 
% Therefore, each cell of the first line of a table is formatted as follows:
% {\fontsize{11pt}{12pt}\selectfont <content here> }

%  -- FHBgld paragraphs etc.
\setlength{\parskip}{6pt} % Paragraph distance 6pt as required by FHBgld
\setlength\parindent{0pt} % No indent as required by FHBgld
\setlength{\footskip}{2.5\baselineskip} % Move Textbody close to footer as required by FHBgld
\renewcommand*{\chapterheadstartvskip}{\vspace*{0cm}} % FHBgld starts chapter aligned to top


\RedeclareSectionCommand[beforeskip=0pt,afterskip=.5\baselineskip plus .1\baselineskip minus .167\baselineskip]{chapter}
\RedeclareSectionCommand[beforeskip=-1\baselineskip,afterskip=.2\baselineskip]{section}
\RedeclareSectionCommand[beforeskip=-.75\baselineskip,afterskip=.5\baselineskip]{subsection}
\RedeclareSectionCommand[beforeskip=-.75\baselineskip,afterskip=.25\baselineskip]{subsubsection}
\RedeclareSectionCommand[beforeskip=.5\baselineskip,afterskip=-1em]{paragraph}
\RedeclareSectionCommand[beforeskip=-.5\baselineskip,afterskip=-1em]{subparagraph}

% %\renewcommand{\rmdefault}{pplj} % Replaces all Roman fonts with pplj (Palatino-Roman) if \usepackage{palatino}, not mathpazo

% TODO: also move to  thesis-book preamble
\hyphenation{GitOps}
\hyphenation{OpenGitOps}
\hyphenation{Kubernetes}
\hyphenation{Kustomize}

\makeatletter
\newcommand{\setword}[2]{%
  \phantomsection
  #1\def\@currentlabel{\unexpanded{#1}}\label{#2}%
}
\makeatother


% https://www.overleaf.com/learn/latex/LaTeX_Graphics_using_TikZ%3A_A_Tutorial_for_Beginners_(Part_3)%E2%80%94Creating_Flowcharts
\usepackage{tikz}
\usetikzlibrary{shapes.geometric, arrows}

\tikzstyle{startstop} = [rectangle, rounded corners, 
minimum width=3cm, 
minimum height=1cm,
text centered, 
draw=black, 
% fill=red!30]
fill=white]

\tikzstyle{io} = [trapezium, 
trapezium stretches=true, % A later addition
trapezium left angle=70, 
trapezium right angle=110, 
minimum width=3cm, 
minimum height=0.8cm, text centered, 
% draw=black, fill=blue!30]
draw=black, fill=white]

\tikzstyle{process} = [rectangle, 
minimum width=3cm, 
minimum height=0.8cm, 
text centered, 
% text width=3cm, 
draw=black, 
% fill=orange!30]
fill=white]

\tikzstyle{decision} = [diamond, 
minimum width=3cm, 
minimum height=0.8cm, 
text centered, 
draw=black, 
% fill=green!30]
fill=white]
\tikzstyle{arrow} = [thick,->,>=stealth]








\renewcommand{\autodot}{}% Remove all end-of-counter dots


% %%%%%%%%%%%%%%%%%%%%additional packages and macros%%%%%%%%%%%%%%%%%%%%%%%%%%%%%%
%\usepackage{algorithm2e}
%\usepackage[acronym]{glossaries}
%packages and setings to show colored code
%\usepackage{listings}
%\usepackage{xcolor}
%\definecolor{codegreen}{rgb}{0,0.6,0}
%\definecolor{codegray}{rgb}{0.5,0.5,0.5}
%\definecolor{codepurple}{rgb}{0.58,0,0.82}
%\definecolor{backcolour}{rgb}{0.95,0.95,0.92}

\lstdefinestyle{mystyle}{
    backgroundcolor=\color{backcolour},   
    commentstyle=\color{codegreen},
    keywordstyle=\color{magenta},
    numberstyle=\tiny\color{codegray},
    stringstyle=\color{codepurple},
    basicstyle=\ttfamily\footnotesize,
    breakatwhitespace=false,         
    breaklines=true,                 
    captionpos=b,                    
    keepspaces=true,                 
    numbers=left,                    
    numbersep=5pt,                  
    showspaces=false,                
    showstringspaces=false,
    showtabs=false,                  
    tabsize=2
}

\lstset{style=mystyle}

%\usepackage{showframe}


%%%%%%%%%%%%%%%%%%% added to thesis draft %%%%%%%%%%%%%%%%%%%%%%%%%%%%%%%%
\usepackage{acro}
\usepackage[
    urldate=long,		    % default: short, e.g. 08/15/2010
    style=authoryear-icomp,	% Harvard citation style
    sorting=nty,            % this is default: sort by name, title, year
    style=apa,
    backend=biber,
    language=ngerman,
    sortlocale=de-DE,    % set according to your needs
    natbib=true,		    % if you want to use natbib compatible citation commands; do _not_ use package natbib!
    maxbibnames=1000,
]{biblatex}
\addbibresource{references.bib}
