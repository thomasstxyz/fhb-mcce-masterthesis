\chapter{Ergebnisse}

Die Ergebnisse der bisherigen Literaturrecherche zeigen,
dass für die Promotion von Releases zwischen mehreren Umgebungen
keine Standardpraxis existiert. 
Die bestehenden GitOps-Tools liefern
keinerlei native Lösungen dafür.
Als Best Practice wird vorgeschlagen, dass die Promotion-Operationen
- in Form von einfachen Datei-Kopiervorgängen -
an das CI-System ausgelagert werden sollen.
%\bigskip
%
%\noindent
Der Literatur sowie den eigenen Beobachtungen zufolge,
fehlt jegliche Art der Standardisierung,
wenn es um die Definition von mehreren Umgebungen mit GitOps geht.
\bigskip

\noindent
Als primäres Ergebnis dieser Arbeit ist
eine Antwort auf die Forschungsfragen erwartet,
indem eine Softwarelösung als Prototyp vorgestellt wird.
Der entwickelte Prototyp soll optimalerweise als 
modulare Erweiterung zu Kubernetes dienen.
\bigskip

%\noindent
%In einem Laborexperiment soll bewiesen werden, dass
%
%anhand der preisgegebenen Metriken des Kubernetes Workload basierten Stromverbrauchs
%Kubernetes Workloads adaptiv auf geeignete Nodes zugeordnet werden können,
%auf welchen sie am effizientesten laufen.
