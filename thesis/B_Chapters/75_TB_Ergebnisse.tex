\chapter{Ergebnisse}

Die Ergebnisse der bisherigen Literaturrecherche zeigen,
dass bereits wissenschaftliche Untersuchungen zum Thema
der software-basierten dynamischen Stromverbrauchs-Messungs
unternommen wurden. 
Im Bereich VM-basierten Systemen, prozess-basierte Messung, sowie container-basierte Messung
von Stromverbrauch. Dabei wird oft der Gesamtverbrauch der physischen Maschine - primär die 
CPU-Auslastung und deren Stromverbrauch -
mittels Hardware-Indikatoren und Low-Level Schnittstellen, welche die Hardware zur Verfügung stellt, ausgelesen,
und mittels mathematischer Kalkulationen auf kleinere logische Einheiten versucht zuzuordnen.
\bigskip

Als primäres Ergebnis wird die Kubernetes-native Messung des Stromverbrauchs von Kubernetes Workloads
erwartet. Der entwickelte Prototyp soll als modulare Extension zu Kubernetes dienen.
Es sollen die gewonnenen Metriken über den OpenTelemetry-Standard,
für Observability-Werkzeuge als Service Level Indicators (SLI) dienen.
\bigskip

In dem finalen Laborexperiment soll bewiesen werden, dass
anhand der preisgegebenen Metriken des Kubernetes Workload basierten Stromverbrauchs
Kubernetes Workloads adaptiv auf geeignete Nodes ge-scheduled werden können,
auf welchen sie am effizientesten laufen.
