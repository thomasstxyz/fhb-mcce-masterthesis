\chapter{Results}

The results of the literature review so far show,
that there is no standard practice for the promotion of releases between multiple environments in GitOps.
The existing GitOps tools do not provide
a native solutions for this.
As a best practice, it is suggested that the promotion operations
should be outsourced to the CI system
- in the form of simple file copy operations.
According to the literature as well as current observations,
any kind of standardization is missing,
when it comes to defining multiple environments with GitOps
and promoting releases between them.
\bigskip

\noindent
As the primary outcome of this thesis,
an answer to the research questions is expected,
by presenting a software solution as a prototype.
The developed prototype should ideally serve as a 
modular extension to Kubernetes and existing GitOps tooling.
\bigskip

%\noindent
%In einem Laborexperiment soll bewiesen werden, dass
%
%anhand der preisgegebenen Metriken des Kubernetes Workload basierten Stromverbrauchs
%Kubernetes Workloads adaptiv auf geeignete Nodes zugeordnet werden können,
%auf welchen sie am effizientesten laufen.

\noindent
The researched results,
the developed solution approach,
as well as the proposed prototype,
can finally be
be donated to the CNCF.
