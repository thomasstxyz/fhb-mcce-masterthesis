\chapter{Evaluation and Results}
\label{evaluation-and-results}

In the following chapter,
the results of the research will be presented,
and evaluated with a holistic view on the research problem of release promotion in GitOps environments.
The results primarily stem from the designed and developed prototype,
and the learning from the prototyping process.
The conducted interviews with the working professionals also gave several interesting insights into
the problem statement from their point of view.
Next to the presentation of the results, they are also evaluated on how they provide a solution
to the problem.
Especially the functionality implemented in the prototype is evaluated against the research objectives,
which have been defined, and map directly to distinct problem items.

The evaluation of the prototype's functionality against the solution objectives,
was already carried out in section \ref{prototype:evaluation} earlier in the thesis.
It was proven, that for each solution objective defined in section \ref{interviews:definitionSolutionObjectives}
the respective functionality was implemented in the prototype,
and demonstrated in a proof of concept.
It was described, how for each solution objective,
a solution to the respective problem can be observed in the
proof of concept demonstration in section \ref{prototype:demonstration}.

\section*{Feedback for the Prototype}

{\color{red}TODO opinions of working professionals}

%How well does the implemented functionality provide a solution to the research objectives?

The research results are evaluated with a holistic view,
with the focus on the research questions defined in section \ref{introduction:research-question}.

\section*{\ref{RQ1.1}}

A possible solution to this research question was presented in chapter
\ref{chapter:prototype}
by means of describing the design of a prototype
of a Kubernetes operator for handling the operations of promotions for GitOps environments.

Section \ref{prototype:design:async-gitops-deployments} describes the asynchronous nature of
GitOps deployments, and where the proposed operator fits within this architectural pattern.

Section \ref{prototype:design:abstract-models} presents a qualitative description 
of how abstract models for environments and promotions in the context of the operator pattern
could be designed.

\section*{\ref{RQ1.2}}

Described in sections \ref{prototype:design:design-custom-resources}, \ref{prototype:design:mockups-custom-resources} and \ref{prototype:design:alternative-mockups},
a possible implementation of the abstract models is presented.
It is in the form of declarative custom resources for extending the Kubernetes API.

Moreover in section \ref{prototype:design:go-types},
the translations of the custom resources into Go types, which are used in the Kubebuilder framework,
are described.
In section \ref{prototype:design:controller-logic},
the proposed controller logic for the environment, as well as the promotion controller is presented

The proposed prototype is implemented and demonstrated in a proof of concept in section
\ref{prototype:demonstration}.
How well the implemented functionality provides a solution to the research objectives,
is evaluated in section \ref{prototype:evaluation}.

\section*{\ref{RQ1}}

The combined evaluations of the sub research questions provide a possible answer to the
overarching research question for the thesis.

This thesis proposes one possible way of how
the promotion of releases in GitOps environments can be designed.
In no way, shape or form does this concrete research try to
propose a definitive answer or solution to the research question.





































