\chapter{Stand des Wissens / Stand der Technik}

\citeauthor{colmant:hal-01130030} (\citeyear{colmant:hal-01130030})
erforschen "Process-level Power Estimation in VM-based Systems".
Dabei wird die Software BitWatts implementiert,
welches auf dem Python-Framework PowerAPI basiert.
Lösungen auf dem aktuellen Stand der Technik können eine grobkörnige Leistungsabschätzung 
in virtualisierten Umgebungen durchführen, 
wobei virtuelle Maschinen (VMs) in der Regel als Blackbox behandelt werden.
\citeauthor{colmant:hal-01130030} (\citeyear{colmant:hal-01130030})
schlagen eine feinkörnige Überwachungs-Middleware vor, 
die eine genaue Leistungsabschätzung von Softwareprozessen in Echtzeit ermöglicht, 
welche auf jeder Virtualisierungsebene in einem System laufen.
Die Middleware-Implementierung mit dem Namen BitWatts 
baut auf einer verteilten Actor-Implementierung auf, 
um die Prozessnutzung zu erfassen und einen feinkörnigen Stromverbrauch abzuleiten, 
ohne dass Hardware-Investitionen, wie Stromzähler erforderlich sind
\autocite{colmant:hal-01130030}.
\bigskip

\citeauthor{fieni:hal-02470128} (\citeyear{fieni:hal-02470128})
forschen in
"SmartWatts: Self-Calibrating Software-Defined Power Meter for Containers"
über ein leichtgewichtiges Energieüberwachungssystem, das die CPU- und DRAM-Leistungsmodelle 
durch Online-Kalibrierung automatisch anpasst, um die Genauigkeit der Leistungsabschätzung von Containern 
während der Laufzeit zu maximieren. Im Gegensatz zu state-of-the-art Techniken erfordert SmartWatts keine vorherige
Trainingsphase oder Hardware-Ausrüstung zur Konfiguration der Leistungsmodelle und kann daher ohne Kosten 
auf einer Vielzahl von Rechnern eingesetzt werden, einschließlich der neuesten Leistungsoptimierungen
\autocite{fieni:hal-02470128}.
\bigskip

\citeauthor{8705815} (\citeyear{8705815})
beschreiben die komplexe System-of-Systems-Natur von Rechenzentren und diskutieren 
die in der Branche verwendeten Servicemodelle. Sie beschreiben ein ganzheitliches Scheduling-System, 
das den Standard-Scheduler im Kubernetes-Container-System ersetzt 
und sowohl Software- als auch Hardware-Modelle berücksichtigt. 
Sie erörtern die ersten Ergebnisse des Einsatzes dieses Schemas in einem realen Rechenzentrum, 
in dem eine Reduzierung des Stromverbrauchs um 10-20 \% beobachtet wird. 
Es wird gezeigt, dass ein intelligenter Scheduler durch die Einführung von Hardwaremodellen 
in eine softwarebasierte Lösung die Effizienz von Rechenzentren erheblich verbessern kann
\autocite{8705815}.



