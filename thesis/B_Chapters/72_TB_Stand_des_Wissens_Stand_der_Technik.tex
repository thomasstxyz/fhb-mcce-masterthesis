\chapter{Stand des Wissens / Stand der Technik}

% TODO: what not to do for gitops environments (multi-branch, etc.)
% TODO: https://codefresh.io/blog/stop-using-branches-deploying-different-gitops-environments/

In einem Blogbeitrag
"How to Model Your Gitops Environments and Promote Releases between Them"
liefert
\citeauthor{codefreshHowToModelGitOpsEnvironmentsAndPromote} (\citeyear{codefreshHowToModelGitOpsEnvironmentsAndPromote})
von Codefresh
Best Practices zu der Problemstellung.
Codefresh ist die Organisation hinter dem Argo Project
% TODO: cite Argo Project
\autocite{argoProjWebsite},
welches eine Suite an Open-Source-Tools für GitOps
entwickelt
\autocite{codefreshHowToModelGitOpsEnvironmentsAndPromote}.

\citeauthor{codefreshHowToModelGitOpsEnvironmentsAndPromote} (\citeyear{codefreshHowToModelGitOpsEnvironmentsAndPromote})
schlägt folgende Kategorien der "Umgebungskonfiguration" vor:

\begin{itemize}
    \item Die \textbf{App-Version} in Form des verwendeten Container-Tags. Dies ist wahrscheinlich die wichtigste Einstellung in einem Kubernetes-Manifest (soweit es um die Förderung der Umgebung geht). Je nach Anwendungsfall können Sie sich damit begnügen, die Version des Container-Images zu ändern. In vielen Fällen erfordert jedoch eine neue Änderung im Quellcode auch eine neue Änderung in der Bereitstellungsumgebung
    \item Kubernetes-spezifische Einstellungen für Ihre Anwendung. Dazu gehören die Replikate der Anwendung und andere Kubernetes-bezogene Informationen wie Ressourcenlimits, Zustandsprüfungen, persistente Volumes, Affinitätsregeln usw.
    \item Hauptsächlich statische Geschäftseinstellungen. Hierbei handelt es sich um Einstellungen, die nichts mit Kubernetes zu tun haben, sondern mit dem Geschäft Ihrer Anwendung. Dabei kann es sich um externe URLs, interne Warteschlangengrößen, UI-Standardwerte, Authentifizierungsprofile usw. handeln. Mit "größtenteils statisch" meine ich Einstellungen, die einmal für jede Umgebung definiert werden und sich dann nie mehr ändern. Sie möchten zum Beispiel, dass Ihre Produktionsumgebung immer production.paypal.com und Ihre Nicht-Produktionsumgebung staging.paypal.com verwendet. Diese Einstellung sollten Sie niemals zwischen Umgebungen verschieben.
    \item Nicht-statische Geschäftseinstellungen. Dies ist dasselbe wie der vorige Punkt, aber es umfasst Einstellungen, die Sie zwischen Umgebungen übertragen möchten. Dabei kann es sich um eine globale Mehrwertsteuer-Einstellung, Ihre Parameter für die Empfehlungsmaschine, die verfügbaren Bitraten-Codierungen und jede andere Einstellung handeln, die für Ihr Unternehmen spezifisch ist.
\end{itemize}




\citeauthor{colmant:hal-01130030} (\citeyear{colmant:hal-01130030})
erforschen "Process-level Power Estimation in VM-based Systems".
Dabei wird die Software BitWatts implementiert,
welches auf dem Python-Framework PowerAPI basiert.
Lösungen auf dem aktuellen Stand der Technik können eine grobkörnige Leistungsabschätzung 
in virtualisierten Umgebungen durchführen, 
wobei virtuelle Maschinen (VMs) in der Regel als Blackbox behandelt werden.
\citeauthor{colmant:hal-01130030} (\citeyear{colmant:hal-01130030})
schlagen eine feinkörnige Überwachungs-Middleware vor, 
die eine genaue Leistungsabschätzung von Softwareprozessen in Echtzeit ermöglicht, 
welche auf jeder Virtualisierungsebene in einem System laufen.
Die Middleware-Implementierung mit dem Namen BitWatts 
baut auf einer verteilten Actor-Implementierung auf, 
um die Prozessnutzung zu erfassen und einen feinkörnigen Stromverbrauch abzuleiten, 
ohne dass Hardware-Investitionen, wie Stromzähler erforderlich sind
\autocite{colmant:hal-01130030}.
\bigskip

\citeauthor{fieni:hal-02470128} (\citeyear{fieni:hal-02470128})
forschen in
"SmartWatts: Self-Calibrating Software-Defined Power Meter for Containers"
über ein leichtgewichtiges Energieüberwachungssystem, das die CPU- und DRAM-Leistungsmodelle 
durch Online-Kalibrierung automatisch anpasst, um die Genauigkeit der Leistungsabschätzung von Containern 
während der Laufzeit zu maximieren. Im Gegensatz zu state-of-the-art Techniken erfordert SmartWatts keine vorherige
Trainingsphase oder Hardware-Ausrüstung zur Konfiguration der Leistungsmodelle und kann daher ohne Kosten 
auf einer Vielzahl von Rechnern eingesetzt werden, einschließlich der neuesten Leistungsoptimierungen
\autocite{fieni:hal-02470128}.
\bigskip

\citeauthor{8705815} (\citeyear{8705815})
beschreiben die komplexe System-of-Systems-Natur von Rechenzentren und diskutieren 
die in der Branche verwendeten Servicemodelle. Sie beschreiben ein ganzheitliches Scheduling-System, 
das den Standard-Scheduler im Kubernetes-Container-System ersetzt 
und sowohl Software- als auch Hardware-Modelle berücksichtigt. 
Sie erörtern die ersten Ergebnisse des Einsatzes dieses Schemas in einem realen Rechenzentrum, 
in dem eine Reduzierung des Stromverbrauchs um 10-20 \% beobachtet wird. 
Es wird gezeigt, dass ein intelligenter Scheduler durch die Einführung von Hardwaremodellen 
in eine softwarebasierte Lösung die Effizienz von Rechenzentren erheblich verbessern kann
\autocite{8705815}.







