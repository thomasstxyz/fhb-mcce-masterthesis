\chapter{Fundamentals} 	% Produces section heading.  Lower-level

Basics such as theory, definitions, relevant theories, related work and state of the art should be included here.

TODO

\section{General definitions}

The following section includes general definitions of terms on the topic.

\subsection*{GitOps}

This thesis aims at adhering to the definition of the term GitOps,
%for the term GitOps, its principles and the glossary around it.
as specified by the CNCF project OpenGitOps.
The overall goal of OpenGitOps is to establish a clear vendor-neutral,
principle-driven meaning of GitOps,
which shall provide a foundation for interoperability between tools, conformance and certification through enduring programs, documents and code
\autocite{opengitopsDocuments}.

\noindent
The primary four principles
for the desired state of a
system managed by GitOps are the following \autocite{gitopsPrinciplesv100}:

\begin{itemize}
	\item \textbf{Declarative} \\
		A system managed by GitOps must have its desired state expressed declaratively.
	\item \textbf{Versioned and Immutable} \\
		Desired state is stored in a way that enforces immutability, versioning and retains a complete version history.
	\item \textbf{Pulled Automatically} \\
		Software agents automatically pull the desired state declarations from the source.
	\item \textbf{Continuously Reconciled} \\
		Software agents continuously observe actual system state and attempt to apply the desired state.
\end{itemize}

% ensure list does not lap over multiple pages in PDF export

These principles are defined in OpenGitOps version 1.0.0,
along with glossary
\autocite{gitopsGlossary}
for associated terms and concepts.
%The ones needed for understanding of this thesis,
%are the following:

\subsection*{Continuous}
\begin{quotation}
\noindent
"Continuous" is intended to match the industry standard term: reconciliation continues to happen, not that it must be instantaneous.
\autocite{gitopsGlossary}
\end{quotation}

\subsection*{Declarative Description}
\begin{quotation}
\noindent
A configuration that describes the desired operating state of a system without specifying procedures for how that state will be achieved. This separates configuration (the desired state) from the implementation (commands, API calls, scripts etc.) used to achieve that state.
\autocite{gitopsGlossary}
\end{quotation}

\subsection*{Desired State}
\begin{quotation}
\noindent
The aggregate of all configuration data that is sufficient to recreate the system so that instances of the system are behaviourally indistinguishable. This configuration data generally does not include persistent application data, eg. database contents, though often does include credentials for accessing that data, or configuration for data recovery tools running on that system.
\autocite{gitopsGlossary}
\end{quotation}

\subsection*{Drift}
\begin{quotation}
\noindent
When a system's actual state has moved or is in the process of moving away from the desired state, this is often referred to as drift.
\autocite{gitopsGlossary}
\end{quotation}

\subsection*{Reconciliation}
\begin{quotation}
\noindent
The process of ensuring the actual state of a system matches its desired state. Contrary to traditional CI/CD where automation is generally driven by pre-set triggers, in GitOps reconciliation is triggered whenever there is a divergence. Divergence could be due to the actual state unintentionally drifting from the desired state declarations, or a new desired state declaration version having been changed intentionally. Actions are taken based on policies around feedback from the system and previous reconciliation attempts, in order to reduce deviation over time.
\autocite{gitopsGlossary}
\end{quotation}

\subsection*{Software System}
\begin{quotation}
\noindent
A software system managed by GitOps includes:
\begin{itemize}
	\item One or more runtime environments consisting of resources under management
	\item The management agents within each runtime
	\item Policies for controlling access and management of repositories, deployments, runtimes
\end{itemize}
\autocite{gitopsGlossary}
\end{quotation}

\subsection*{State Store}
\begin{quotation}
\noindent
A system for storing immutable versions of desired state declarations. This state store should provide access control and auditing on the changes to the Desired State. Git, from which GitOps derives its name, is the canonical example used as this state store but any other system that meets these criteria may be used. In all cases, these state stores must be properly configured and precautions must be taken to comply with requirements set out in the GitOps Principles.
\autocite{gitopsGlossary}
\end{quotation}

\subsection*{Feedback}
\begin{quotation}
\noindent
Open GitOps follows control-theory and operates in a closed-loop. In control theory, feedback represents how previous attempts to apply a desired state have affected the actual state. For example if the desired state requires more resources than exist in a system, the software agent may make attempts to add resources, to automatically rollback to a previous version, or to send alerts to human operators.
\autocite{gitopsGlossary}
\end{quotation}


%which is a Cloud Native Computing Foundation (CNCF) sandbox project
%in the App Delivery TAG 
%under the GitOps Working Group.

%However, the research is not limited to the definitions in OpenGitOps,
%and might even suggest changes.








\section{Related work / state of research}


Since the introduction of the term GitOps by
\citeauthor{gitopsOperationsByPullRequest2017Weaveworks} (\citeyear{gitopsOperationsByPullRequest2017Weaveworks})
at Weaveworks,
there has been
insufficient
%limited
scientific literature written about the topic.
The following paragraph highlights prior 
scientific research done on the subject "GitOps",
which has mainly been targeted at
evaluating the concept of GitOps in general.
\bigskip

%as opposed to focusing on new and specific problems that emerge with the adoption of GitOps,
%like the one focused on in this thesis.
%\bigskip

\noindent
\citeauthor{limoncelli_gitopsPathToMoreSelfService} (\citeyear{limoncelli_gitopsPathToMoreSelfService})
draws attention to GitOps in an article, where the concept is brought to the reader in an easily digestible way.
The author clearly highlights the benefits of adoption of GitOps,
however does not mention any negative aspects.
%\citeauthor{whatIsGitOpsSalecha2023} (\citeyear{whatIsGitOpsSalecha2023})
%discusses the definition of the "GitOps"-term
%\autocite{whatIsGitOpsSalecha2023}.
\citeauthor{gitopsTheEvolutionOfDevops9565152} (\citeyear{gitopsTheEvolutionOfDevops9565152})
discuss the idea of GitOps as an evolution of DevOps. They conduct research on the definition of both terms.
\citeauthor{continuousDeploymentIOTEdgeComputingAGitOpsImplementation_9820108} (\citeyear{continuousDeploymentIOTEdgeComputingAGitOpsImplementation_9820108})
publish a conference paper about implementing GitOps in Internet of Things (IoT) Edge Computing
to achieve Continuous Deployment.
They present a proof of concept to check the feasibility of applying GitOps
in IoT Edge Computing solutions.
\citeauthor{analysisOnDeclarativePullBasedDeploymentGitOpsArgoCD_9563984} (\citeyear{analysisOnDeclarativePullBasedDeploymentGitOpsArgoCD_9563984})
present an analysis of declarative and pull-based deployment models
following GitOps principles by using the tool ArgoCD;
and focuses on the advantages compared to push-based deployment models.
\bigskip

\noindent
Hitherto, there have not been any publications
in academic journals or conferences
on the specific topic of
modeling multiple deployment environments with GitOps and promoting releases between them.
However,
%\bigskip
%
%\noindent
\citeauthor{kostisKapelonisMeetACodefresher} (\citeyear{kostisKapelonisMeetACodefresher})
at Codefresh
published a couple of blog posts in recent years,
where some best practices are presented on the topic.
Codefresh is the organization behind the Argo Project \autocite{argoProjWebsite},
and therefore a major driving force in the GitOps ecosystem.
\bigskip

\noindent
%In
%\citetitle{codefreshStopUsingBranchesGitOpsEnvironments}
\citeauthor{codefreshStopUsingBranchesGitOpsEnvironments} (\citeyear{codefreshStopUsingBranchesGitOpsEnvironments})
discusses
the idea of
modeling different deployment environments by using Git branches.
He
explains thoroughly why this approach is an anti-pattern and should not be used
\autocite{codefreshStopUsingBranchesGitOpsEnvironments}.
%In
%\citetitle{codefreshHowToModelGitOpsEnvironmentsAndPromote}
\citeauthor{codefreshHowToModelGitOpsEnvironmentsAndPromote} (\citeyear{codefreshHowToModelGitOpsEnvironmentsAndPromote})
shares
a multitude of suggestions and best practices
about modeling environments and promoting releases between them.
Different environments are modeled by customizing \autocite{kustomizeIoWebsite} configuration 
in separate files and folders or Git repositories.
For promoting between environments, basic file copy operations are suggested.
It is noted, that these simple file copy operations can easily be automated by an external system,
like a CI/CD system.
\citeauthor{codefreshHowToModelGitOpsEnvironmentsAndPromote} (\citeyear{codefreshHowToModelGitOpsEnvironmentsAndPromote})
suggests four categories of environment configuration.
The application version,
Kubernetes specific settings,
mostly static business settings,
and 
non-static business settings.
While the application version and non-static business settings are promoted,
Kubernetes specific settings and mostly static business settings are generally not promoted between environments
\autocite{codefreshHowToModelGitOpsEnvironmentsAndPromote}.
\bigskip

\noindent
In 2021 the
Argo Project \autocite{argoProjWebsite}
presented the
\citetitle{argocdAutopilotWebsite}-tool \autocite{argocdAutopilotWebsite}
to help new GitOps adopters with
structuring their Git repositories,
and
promoting applications between environments.
This command line interface (CLI) tool,
which includes some of the earlier presented best practices,
helps with the initial bootstrap process for ArgoCD.
\bigskip

% Schreib einen Absatz, der erklärt, inwieweit deine Arbeit anders ist als Related Work. Arbeite das Delta (den Unterschied) zwischen Related Work vs. Your Work aus und betone nochmal deinen Beitrag zum GitOps-Forschungsfeld (das was du beitragen wirst im Vergleich zu allen anderen)

\noindent
Prior research on the concrete problem is focused on presenting
best practices and suggestions
which users need to manually implement themselves.
In addition it is suggested to let an external CI/CD system handle the promotion process.
Conversely, this thesis will bring forward
abstract models of environments and promotion processes,
which are implemented in the proposed prototype tooling.
The prototype will assess the feasibility of
defining deployment environments and promotion processes declaratively,
following the GitOps principles.






%The following are current practices
%shown on the topic,
%how best to model different deployment environments with GitOps.
%
%
%\section{branch-per-environment approach}
%
%\citeauthor{codefreshStopUsingBranchesGitOpsEnvironments} (\citeyear{codefreshStopUsingBranchesGitOpsEnvironments})
%published Codefresh's best practices on modeling environments in GitOps 
%in the blog post "Stop Using Branches for Deploying to Different GitOps Environments".
%The problems with the branch-per-environment approach are shown.
%This approach is an anti-pattern,
%and to be avoided at all costs.
%A number of points are made and justified in detail.
%
%\begin{itemize}
%	\item Using different Git branches for deployment environments is a relic of the past.
%	\item Pull requests and merges between different branches is problematic.
%	\item People are tempted to include environment specific code and create configuration drift.
%	\item As soon as you have a large number of environments, maintenance of all environments gets quickly out of hand.
%	\item The branch-per-environment model goes against the existing Kubernetes ecosystem.
%\end{itemize}
%
%\noindent
%\autocite{codefreshStopUsingBranchesGitOpsEnvironments}
%
%% immer fließtext nach aufzählungen, tabellen, etc.
%
%\section{folder-per-environment approach}
%
%\citeauthor{codefreshHowToModelGitOpsEnvironmentsAndPromote} (\citeyear{codefreshHowToModelGitOpsEnvironmentsAndPromote})
%provides some best practices from Codefresh on the problem in the blog post: "How to Model Your Gitops Environments and Promote Releases Between Them."
%%\bigskip
%%
%%\noindent
%Es werden folgende Kategorien der "Umgebungskonfiguration" vorgeschlagen:
%
%\begin{itemize}
%    \item \textbf{Application version} in the form of the container tag used. This is the setting that increases with each release and decreases with each rollback - and is always promoted between environments.
%	\item \textbf{Kubernetes-specific settings} for the application (manifests).
%	\item \textbf{Mainly static settings} for the business logic. These include external URLs, internal queue sizes, UI default values, authentication profiles, etc. By "mostly static" they mean settings that are defined once for each environment and then generally do not change. These settings should never be promoted between environments.
%	\item \textbf{Non-Static Business Settings}. This is the same as the previous item, but it includes settings that are indeed promoted between environments. These can be a global VAT setting, parameters for the recommendation engine, the available bitrate encodings, and any other setting that is specific to the business or enterprise.
%\end{itemize}
%
%\noindent
%\autocite{codefreshHowToModelGitOpsEnvironmentsAndPromote}
%\bigskip
%
%\noindent
%The application version and non-static business settings are promoted between environments.
%\citeauthor{codefreshHowToModelGitOpsEnvironmentsAndPromote} (\citeyear{codefreshHowToModelGitOpsEnvironmentsAndPromote})
%suggests defining the configuration parameters per category in a separate file.
%As a result, the four categories can be promoted independently.
%The folder-per-environment approach is proposed.
%% visuell darstellen die 4 config files ?
%In general, all promotions are just copy operations. 
%Unlike the branch-per-environment approach, with the folder-per-environment approach 
%everything can be moved from any environment to any other environment
%without fear that the wrong changes will be applied. 
%Especially when it comes to backporting configurations, the folder-per-environment approach shines, 
%because configurations can be easily moved both "forward" and "backward", 
%even between unrelated environments.
%\bigskip
%
%\noindent
%\textbf{Scenario}: Promotion of the application version from qa to staging-us.
%
%\begin{verbatim}
%	cp envs/qa/version.yml envs/staging-us/version.yml
%\end{verbatim}
%
%\noindent
%\textbf{Scenario}: Promotion of the application from prod-eu to prod-us together with the additional configuration. The settings file(s) are also copied.
%
%\begin{verbatim}
%	cp envs/prod-eu/version.yml envs/prod-us/version.yml
%	cp envs/prod-eu/settings.yml envs/prod-us/settings.yml
%\end{verbatim}
%
%\noindent
%\textbf{Scenario}: Backport all settings from qa to integration-non-gpu.
%
%\begin{verbatim}
%	cp envs/qa/settings.yml envs/integration-non-gpu/settings.yml
%\end{verbatim}
%
%\noindent
%\citeauthor{codefreshHowToModelGitOpsEnvironmentsAndPromote} (\citeyear{codefreshHowToModelGitOpsEnvironmentsAndPromote})
%notes,
%that the copy operations are for illustrative purposes only. 
%In a real-world application, this operation would be performed automatically by a CI system 
%or other orchestration tool. And depending on the environment 
%perhaps a pull request should be created first instead of processing the folder in the main branch directly
%\autocite{codefreshHowToModelGitOpsEnvironmentsAndPromote}.
%\bigskip
%
%\noindent
%\citeauthor{codefreshHowToModelGitOpsEnvironmentsAndPromote} (\citeyear{codefreshHowToModelGitOpsEnvironmentsAndPromote})
%suggests the following advantages of the folder-per-environment approach:
%
%\begin{itemize}
%	\item The order of commits in the Git repository is irrelevant. When a file is copied from one folder to the next, the commit history doesn't matter.
%	\item If only files are copied, only exactly what is needed is promoted, and nothing else.
%	\item No need to use git cherry picks or other advanced git methods to promote releases. Only files need to be copied.
%	\item The user is free to commit any change from any environment to any environment, without any restrictions on the proper "order" of environments.
%	\item Using diff operations on files, it is easy to track what is different between environments in all directions (both in the source and target environments and vice versa).
%\end{itemize}
%
%\noindent
%\autocite{codefreshHowToModelGitOpsEnvironmentsAndPromote}

























\section{Instruction included in the original FHBgld word processor template}
\subsection{General definitions}
Die in dieser Formatvorlage beispielhaft enthaltenen Überschriften sind auf die im
konkreten Fall tatsächlich passenden Überschriften anzupassen.
In diesem Teil der Arbeit werden die zum eindeutigen Verständnis unbedingt
erforderlichen Grundlagen und Definitionen sowie die Erklärung wichtiger Begriffe
angeführt.
Die Gliederungspunkte müssen möglichst prägnant bezeichnet werden.
\subsection{Related work / state of research}
Auch die neuesten Entwicklungen und Arbeiten auf diesem Gebiet (Stand der
Wissenschaft oder auch state-of-the-art) sind darzulegen, wobei diese je nach Thema
auch in der 1. Gliederungsebene behandelt werden können.

\section{Ordinary text}
% A '%' character causes TeX to ignore all remaining text on the line,
% and is used for comments like this one.

% sections are begun with similar 
% \subsection and \subsubsection commands.

The ends  of words and sentences are marked by spaces. It doesn't matter how many 
spaces    you type; one is as good as 100.  The
end of   a line counts as a space.

One   or more   blank lines denote the  end 
of  a paragraph.  

Since any number of consecutive spaces are treated
like a single one, the formatting of the input
file makes no difference to
\LaTeX,                % The \LaTeX command generates the LaTeX logo.
but it makes a difference to you.  When you use 
\LaTeX \cite{lamport94},  % \cite inserts a reference, which you define at the end of the document
making your input file as easy to read
as possible will be a great help as you write 
your document and when you change it.  This sample 
file shows how you can add comments to your own input 
file.

Because printing is different from typewriting,
there are a number of things that you have to do
differently when preparing an input file than if
you were just typing the document directly.
Quotation marks like
``this'' 
have to be handled specially, as do quotes within
quotes:
``\,`this'            % \, separates the double and single quote.
is what I just 
wrote, not  `that'\,''.  

Dashes come in three sizes: an 
intra-word 
dash, a medium dash for number ranges like 
1--2, 
and a punctuation 
dash---like 
this.

A sentence-ending space should be larger than the
space between words within a sentence.  You
sometimes have to type special commands in
conjunction with punctuation characters to get
this right, as in the following sentence.
Gnats, gnus, etc.\ all  % `\ ' makes an inter-word space.
begin with G\@.         % \@ marks end-of-sentence punctuation.
You should check the spaces after periods when
reading your output to make sure you haven't
forgotten any special cases.  Generating an
ellipsis
\ldots\               % `\ ' is needed after `\ldots' because TeX 
% ignores spaces after command names like \ldots 
% made from \ + letters.
%
% Note how a `%' character causes TeX to ignore 
% the end of the input line, so these blank lines 
% do not start a new paragraph.
%
with the right spacing around the periods requires
a special command.

\LaTeX\ interprets some common characters as
commands, so you must type special commands to
generate them.  These characters include the
following:
\$ \& \% \# \{ and \}.

In printing, text is usually emphasized with an
\emph{italic}  
type style.  

\begin{em}
	A long segment of text can also be emphasized 
	in this way.  Text within such a segment can be 
	given \emph{additional} emphasis.
\end{em}

It is sometimes necessary to prevent \LaTeX\ from
breaking a line where it might otherwise do so.
This may be at a space, as between the ``Mr.''\ and
``Jones'' in
``Mr.~Jones'',        % ~ produces an unbreakable interword space.
or within a word---especially when the word is a
symbol like
\mbox{\emph{itemnum}} 
that makes little sense when hyphenated across
lines.

Footnotes\footnote{This is an example of a footnote.}
pose no problem.

\LaTeX\ is good at typesetting mathematical formulas
like
\( x-3y + z = 7 \) 
or
\( a_{1} > x^{2n} + y^{2n} > x' \)
or  
\( AB  = \sum_{i} a_{i} b_{i} \).
The spaces you type in a formula are 
ignored.  Remember that a letter like
$x$                   % $ ... $  and  \( ... \)  are equivalent
is a formula when it denotes a mathematical
symbol, and it should be typed as one.
Furthermore you can add a formula as Images or Tables, see Formula  \hyperref[eq:abc]{\ref{eq:abc}}
\begin{equation}
	\label{eq:abc}
	a+b=c
\end{equation}

It is sometimes necessary to prevent \LaTeX\ from
breaking a line where it might otherwise do so.
This may be at a space, as between the ``Mr.''\ and
``Jones'' in
``Mr.~Jones'',        % ~ produces an unbreakable interword space.
or within a word---especially when the word is a
symbol like
\mbox{\emph{itemnum}} 
that makes little sense when hyphenated across
lines.
