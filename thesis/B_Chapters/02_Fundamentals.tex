\chapter{Theoretical Background} 	% Produces section heading.  Lower-level

Basics such as theory, definitions, relevant theories, related work and state of the art should be included here.

TODO

% TODO: maximal 1.1.1 
% TODO: wenns 1.1 gibt, muss es auch 1.2 geben

\section{General Definitions}

The following section includes general definitions of terms on the topic.

\subsection*{GitOps}

This thesis aims at adhering to the definition of the term GitOps,
%for the term GitOps, its principles and the glossary around it.
as specified by the CNCF project OpenGitOps.
The overall goal of OpenGitOps is to establish a clear vendor-neutral,
principle-driven meaning of GitOps,
which shall provide a foundation for interoperability between tools, conformance and certification through enduring programs, documents and code
\autocite{opengitopsDocuments}.

\noindent
The primary four principles
for the desired state of a
system managed by GitOps are the following \autocite{gitopsPrinciplesv100}:

\begin{itemize}
	\item \textbf{Declarative} \\
		A system managed by GitOps must have its desired state expressed declaratively.
	\item \textbf{Versioned and Immutable} \\
		Desired state is stored in a way that enforces immutability, versioning and retains a complete version history.
	\item \textbf{Pulled Automatically} \\
		Software agents automatically pull the desired state declarations from the source.
	\item \textbf{Continuously Reconciled} \\
		Software agents continuously observe actual system state and attempt to apply the desired state.
\end{itemize}

% ensure list does not lap over multiple pages in PDF export

These principles are defined in OpenGitOps version 1.0.0,
along with glossary
\autocite{gitopsGlossary}
for associated terms and concepts.
%The ones needed for understanding of this thesis,
%are the following:

\subsection*{Continuous}
\begin{quotation}
\noindent
"Continuous" is intended to match the industry standard term: reconciliation continues to happen, not that it must be instantaneous.
\autocite{gitopsGlossary}
\end{quotation}

\subsection*{Declarative Description}
\begin{quotation}
\noindent
A configuration that describes the desired operating state of a system without specifying procedures for how that state will be achieved. This separates configuration (the desired state) from the implementation (commands, API calls, scripts etc.) used to achieve that state.
\autocite{gitopsGlossary}
\end{quotation}

\subsection*{Desired State}
\begin{quotation}
\noindent
The aggregate of all configuration data that is sufficient to recreate the system so that instances of the system are behaviourally indistinguishable. This configuration data generally does not include persistent application data, eg. database contents, though often does include credentials for accessing that data, or configuration for data recovery tools running on that system.
\autocite{gitopsGlossary}
\end{quotation}

\subsection*{Drift}
\begin{quotation}
\noindent
When a system's actual state has moved or is in the process of moving away from the desired state, this is often referred to as drift.
\autocite{gitopsGlossary}
\end{quotation}

\subsection*{Reconciliation}
\begin{quotation}
\noindent
The process of ensuring the actual state of a system matches its desired state. Contrary to traditional CI/CD where automation is generally driven by pre-set triggers, in GitOps reconciliation is triggered whenever there is a divergence. Divergence could be due to the actual state unintentionally drifting from the desired state declarations, or a new desired state declaration version having been changed intentionally. Actions are taken based on policies around feedback from the system and previous reconciliation attempts, in order to reduce deviation over time.
\autocite{gitopsGlossary}
\end{quotation}

\subsection*{Software System}
\begin{quotation}
\noindent
A software system managed by GitOps includes:
\begin{itemize}
	\item One or more runtime environments consisting of resources under management
	\item The management agents within each runtime
	\item Policies for controlling access and management of repositories, deployments, runtimes
\end{itemize}
\autocite{gitopsGlossary}
\end{quotation}

\subsection*{State Store}
\begin{quotation}
\noindent
A system for storing immutable versions of desired state declarations. This state store should provide access control and auditing on the changes to the Desired State. Git, from which GitOps derives its name, is the canonical example used as this state store but any other system that meets these criteria may be used. In all cases, these state stores must be properly configured and precautions must be taken to comply with requirements set out in the GitOps Principles.
\autocite{gitopsGlossary}
\end{quotation}

\subsection*{Feedback}
\begin{quotation}
\noindent
Open GitOps follows control-theory and operates in a closed-loop. In control theory, feedback represents how previous attempts to apply a desired state have affected the actual state. For example if the desired state requires more resources than exist in a system, the software agent may make attempts to add resources, to automatically rollback to a previous version, or to send alerts to human operators.
\autocite{gitopsGlossary}
\end{quotation}


%which is a Cloud Native Computing Foundation (CNCF) sandbox project
%in the App Delivery TAG 
%under the GitOps Working Group.

%However, the research is not limited to the definitions in OpenGitOps,
%and might even suggest changes.




\section{Extending Kubernetes}
\url{https://kubernetes.io/docs/concepts/extend-kubernetes/}

\section{Operator Pattern}
\url{https://kubernetes.io/docs/concepts/extend-kubernetes/operator/}

\section{Kubebuilder Framework}
\url{https://github.com/kubernetes-sigs/kubebuilder}



\section{Promotion with push-based CI/CD Pipelines}

\section{Purpose of multiple environments}

\section{Trend towards Progressive Delivery}

\section{Long-living environments - past and present}




















%
%\section{Instruction included in the original FHBgld word processor template}
%\subsection{General definitions}
%Die in dieser Formatvorlage beispielhaft enthaltenen Überschriften sind auf die im
%konkreten Fall tatsächlich passenden Überschriften anzupassen.
%In diesem Teil der Arbeit werden die zum eindeutigen Verständnis unbedingt
%erforderlichen Grundlagen und Definitionen sowie die Erklärung wichtiger Begriffe
%angeführt.
%Die Gliederungspunkte müssen möglichst prägnant bezeichnet werden.
%\subsection{Related work / state of research}
%Auch die neuesten Entwicklungen und Arbeiten auf diesem Gebiet (Stand der
%Wissenschaft oder auch state-of-the-art) sind darzulegen, wobei diese je nach Thema
%auch in der 1. Gliederungsebene behandelt werden können.
%
%\section{Ordinary text}
%% A '%' character causes TeX to ignore all remaining text on the line,
%% and is used for comments like this one.
%
%% sections are begun with similar 
%% \subsection and \subsubsection commands.
%
%The ends  of words and sentences are marked by spaces. It doesn't matter how many 
%spaces    you type; one is as good as 100.  The
%end of   a line counts as a space.
%
%One   or more   blank lines denote the  end 
%of  a paragraph.  
%
%Since any number of consecutive spaces are treated
%like a single one, the formatting of the input
%file makes no difference to
%\LaTeX,                % The \LaTeX command generates the LaTeX logo.
%but it makes a difference to you.  When you use 
%\LaTeX \cite{lamport94},  % \cite inserts a reference, which you define at the end of the document
%making your input file as easy to read
%as possible will be a great help as you write 
%your document and when you change it.  This sample 
%file shows how you can add comments to your own input 
%file.
%
%Because printing is different from typewriting,
%there are a number of things that you have to do
%differently when preparing an input file than if
%you were just typing the document directly.
%Quotation marks like
%``this'' 
%have to be handled specially, as do quotes within
%quotes:
%``\,`this'            % \, separates the double and single quote.
%is what I just 
%wrote, not  `that'\,''.  
%
%Dashes come in three sizes: an 
%intra-word 
%dash, a medium dash for number ranges like 
%1--2, 
%and a punctuation 
%dash---like 
%this.
%
%A sentence-ending space should be larger than the
%space between words within a sentence.  You
%sometimes have to type special commands in
%conjunction with punctuation characters to get
%this right, as in the following sentence.
%Gnats, gnus, etc.\ all  % `\ ' makes an inter-word space.
%begin with G\@.         % \@ marks end-of-sentence punctuation.
%You should check the spaces after periods when
%reading your output to make sure you haven't
%forgotten any special cases.  Generating an
%ellipsis
%\ldots\               % `\ ' is needed after `\ldots' because TeX 
%% ignores spaces after command names like \ldots 
%% made from \ + letters.
%%
%% Note how a `%' character causes TeX to ignore 
%% the end of the input line, so these blank lines 
%% do not start a new paragraph.
%%
%with the right spacing around the periods requires
%a special command.
%
%\LaTeX\ interprets some common characters as
%commands, so you must type special commands to
%generate them.  These characters include the
%following:
%\$ \& \% \# \{ and \}.
%
%In printing, text is usually emphasized with an
%\emph{italic}  
%type style.  
%
%\begin{em}
%	A long segment of text can also be emphasized 
%	in this way.  Text within such a segment can be 
%	given \emph{additional} emphasis.
%\end{em}
%
%It is sometimes necessary to prevent \LaTeX\ from
%breaking a line where it might otherwise do so.
%This may be at a space, as between the ``Mr.''\ and
%``Jones'' in
%``Mr.~Jones'',        % ~ produces an unbreakable interword space.
%or within a word---especially when the word is a
%symbol like
%\mbox{\emph{itemnum}} 
%that makes little sense when hyphenated across
%lines.
%
%Footnotes\footnote{This is an example of a footnote.}
%pose no problem.
%
%\LaTeX\ is good at typesetting mathematical formulas
%like
%\( x-3y + z = 7 \) 
%or
%\( a_{1} > x^{2n} + y^{2n} > x' \)
%or  
%\( AB  = \sum_{i} a_{i} b_{i} \).
%The spaces you type in a formula are 
%ignored.  Remember that a letter like
%$x$                   % $ ... $  and  \( ... \)  are equivalent
%is a formula when it denotes a mathematical
%symbol, and it should be typed as one.
%Furthermore you can add a formula as Images or Tables, see Formula  \hyperref[eq:abc]{\ref{eq:abc}}
%\begin{equation}
%	\label{eq:abc}
%	a+b=c
%\end{equation}
%
%It is sometimes necessary to prevent \LaTeX\ from
%breaking a line where it might otherwise do so.
%This may be at a space, as between the ``Mr.''\ and
%``Jones'' in
%``Mr.~Jones'',        % ~ produces an unbreakable interword space.
%or within a word---especially when the word is a
%symbol like
%\mbox{\emph{itemnum}} 
%that makes little sense when hyphenated across
%lines.
