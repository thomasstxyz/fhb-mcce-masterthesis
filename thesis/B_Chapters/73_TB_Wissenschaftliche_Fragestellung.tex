\chapter{Wissenschaftliche Fragestellung}

Das Ziel dieser Arbeit ist es, einen Einblick zu gewinnen, 
inwieweit sich die Messung des Stromverbrauchs von Kubernetes Workloads 
nativ in Kubernetes implementieren lässt,
sodass gemessene Metriken auf der Ebene von Kubernetes Workloads 
für weitere Automatisierungen und
Kubernetes-native Integrationen zur Verfügung stehen.
\bigskip

Es soll mit den dynamisch gemessenen Stromverbrauchs-Metriken
ein Prototyp für ein adaptives Scheduling basierend auf Energieeffizienz entwickelt werden.
Der Prototyp soll in einem Laborexperiment mit dem Standard-Kubernetes-Scheduler verglichen werden,
und auf diese Weise zeigen, dass Workloads automatisch auf Nodes ge-scheduled werden können,
auf welchen sie die beste Energieeffizienz aufweisen.
\bigskip

Die konkrete Forschungsfrage lautet:
\bigskip

\textbf{Inwieweit lässt sich der Stromverbrauch von Kubernetes Workloads feststellen
und für native Integrationen dynamisch zur Verfügung stellen?}
\bigskip



