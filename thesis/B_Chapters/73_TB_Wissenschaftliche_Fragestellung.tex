\chapter{Methodology}

The goal of this thesis is to
address the problem of promoting releases between different environments in GitOps.
\bigskip

\noindent
Large organizations in particular usually have many
non-production and production environments
such as: dev, qa, staging-us, staging-eu, production-us, production-eu.
Usually new releases are automatically rolled out in an environment
such as qa - by means of a CI pipeline.
Now the task is to promote
new changes, which are brought about by a new release,
into subsequent or other environments.
Current GitOps tools do not have a simple answer to
the question about what is the right approach for the process of promotion.
\bigskip

\noindent
In particular, this thesis aims to explore
existing strategies for the solution of the problem
with existing tools.
Furthermore, a prototype of a newly proposed strategy will be developed.
Subsequently, the prototype will be tested in a laboratory experiment
and compared to currently existing strategies.
\bigskip

% TODO: mein Lösungsvorschlag?


\noindent
To achieve the goal of the thesis, the following research questions were identified:

\begin{itemize}
	\item How can the promotion of releases in GitOps environments be designed?
	\begin{itemize}
		\item What possibilities do existing tools offer for the promotion of releases with multiple deployment environments?
		\item How can deployment environments, as well as promotion workflows be modeled abstractly?
		\item How can abstract modeling be used to implement a standardized solution for promoting releases according to GitOps principles?
	\end{itemize}
\end{itemize}

