\chapter{Research Question}




% \section{Research Questions}
% \label{introduction:research-question}






%In this regard a prototype will be developed
%and its functionality demonstrated in a proof-of-concept evaluation.
%The goal for the prototype is to implement it as a modular extension to Kubernetes
%and existing GitOps tooling.
%By doing so, users can promote releases between multiple deployment environments
%following the GitOps principles.
%The main contribution of this thesis is a proof of concept for the developed prototype
%which can be used as an extension to the GitOps toolkit within the CNCF.





% TODO: goal of this reasearch is ...
%The goal of this thesis is to
%address the problem of promoting releases between different environments in GitOps.
%\bigskip
%
%\noindent

The overall goal of the thesis is to
provide a solution to the problem of release promotion in GitOps environments.
The solution shall consist of the abstract design of an operator capable of doing
GitOps-native promotions between environments, as well as its implementation.

Large organizations in particular typically have many
non-production and production environments
such as:
Development (Dev),
Quality Assurance (QA),
Staging-US,
Staging-EU,
Production-US,
Production-EU.
Usually new releases are automatically deployed to an environment,
such as QA, 
by a workflow/pipeline system.
A common task is to promote
new changes, which are introduced by a new release,
into subsequent or other environments.
%
%\noindent
%In particular, this thesis aims to explore
%existing strategies for the solution of the problem
%with existing tools.
%Furthermore, a prototype of a newly proposed strategy will be developed.
%Subsequently, the prototype will be tested in a laboratory experiment
%and compared to currently existing strategies.
%\bigskip
%
%
To achieve the goal of the thesis, the following research questions (RQ) were identified:

\begin{itemize}
	\item \setword{RQ 1: How can the promotion of releases in GitOps environments be designed?}{RQ1}
	\begin{itemize}
		%		\item \setword{RQ 1.1: What possibilities do existing tools offer for the promotion of releases with multiple deployment environments?}{RQ1.1}
		\item \setword{RQ 1.1: How can deployment environments, as well as promotion processes be modeled abstractly?}{RQ1.1}
		\item \setword{RQ 1.2: How can the abstract models be used to implement a standardized solution for promoting releases?}{RQ1.2}
	\end{itemize}
\end{itemize}

% ensure bullet points do not span over page break on final PDF export


