\chapter{Wissenschaftliche Fragestellung}

Das Ziel dieser Arbeit ist es...
\bigskip

\noindent
Um dieses Ziel zu erreichen, wurden folgende Forschungsfragen identifiziert:

\begin{itemize}
	\item ...
\end{itemize}






%In dieser Arbeit soll die Messung des Stromverbrauchs auf Ebene von Kubernetes Workloads erforscht werden.
%Dies würde es beispielsweise erlauben, einen stromverbrauchs-orientierten Kubernetes-nativen
%Scheduler zu implementieren, welcher Nodes für Workloads adaptiv bestimmt,
%je nachdem auf welchen Node-Typen ein bestimmter Workload die beste Energieeffizienz aufweist.
%Node-Typen können sich beispielsweise unterscheiden durch verschiedene
%Prozessor-Architekturen
%\bigskip
%
%Das Ziel dieser Arbeit ist es, einen Einblick zu gewinnen, 
%inwieweit sich die Messung des Stromverbrauchs von Kubernetes Workloads 
%nativ in Kubernetes implementieren lässt,
%sodass gemessene Metriken auf der Ebene von Kubernetes Workloads 
%für weitere Automatisierungen und
%Kubernetes-native Integrationen zur Verfügung stehen.
%\bigskip

%Es soll mit den dynamisch gemessenen Stromverbrauchs-Metriken
%ein Prototyp für ein adaptives Scheduling basierend auf Energieeffizienz entwickelt werden.
%Der Prototyp soll in einem Laborexperiment mit dem Standard-Kubernetes-Scheduler verglichen werden,
%und auf diese Weise zeigen, dass Workloads automatisch Nodes 
%zugeordnet werden können, auf welchen sie die beste Energieeffizienz aufweisen.
%\bigskip

%Die konkrete Forschungsfrage lautet:
%\bigskip
%
%\textbf{Inwieweit lässt sich der Stromverbrauch von Kubernetes Workloads bzw. Applikationen feststellen
%	und ein effizienz-orientiertes Scheduling implementieren?}



