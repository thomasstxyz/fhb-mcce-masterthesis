\chapter{Research Question}

%The goal of this thesis is to
%address the problem of promoting releases between different environments in GitOps.
%\bigskip
%
%\noindent
Large organizations in particular typically have many
non-production and production environments
such as:
Development (Dev),
Quality Assurance (QA),
Staging-US,
Staging-EU,
Production-US,
Production-EU.
Usually new releases are automatically deployed to an environment,
such as QA, 
by a CI/CD system.
Now the task is to promote
new changes, which are brought about by a new release,
into subsequent or other environments.
Current GitOps tools do not have a simple answer to
the question about what is the right approach for the process of promotion.
\bigskip

%\noindent
%In particular, this thesis aims to explore
%existing strategies for the solution of the problem
%with existing tools.
%Furthermore, a prototype of a newly proposed strategy will be developed.
%Subsequently, the prototype will be tested in a laboratory experiment
%and compared to currently existing strategies.
%\bigskip

\noindent
To achieve the goal of the thesis, the following research questions (RQ) were identified:

\begin{itemize}
	\item RQ 1: How can the promotion of releases in GitOps environments be designed?
	\begin{itemize}
		\item RQ 1.1: What possibilities do existing tools offer for the promotion of releases with multiple deployment environments?
		\item RQ 1.2: How can deployment environments, as well as promotion processes be modeled abstractly?
		\item RQ 1.3: How can abstract modeling be used to implement a standardized solution for promoting releases?
	\end{itemize}
\end{itemize}

% TODO: ensure bullet points do not span over page break on final PDF export


