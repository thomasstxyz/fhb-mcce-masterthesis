\chapter{Wissenschaftliche Fragestellung}

Das Ziel dieser Arbeit ist es,
das Problem der Promotion von Releases zwischen verschiedenen Umgebungen in GitOps
zu addressieren.
\bigskip

\noindent
Vor allem große Organisationen verfügen für gewöhnlich über viele
Nicht-Produktions- und Produktions-Umgebungen
wie beispielsweise: dev, qa, staging-us, staging-eu, production-us, production-eu.
In der Regel werden neue Releases automatisch in einer Umgebung
wie beispielsweise qa ausgerollt - mittels einer CI-Pipeline.
Nun ist es die Aufgabe,
neue Änderungen, welche durch ein neues Release herbeigeführt werden,
in nachfolgende bzw. andere Umgebungen zu promoten.
Die aktuellen GitOps-Tools haben keine einfache Antwort auf
die Frage, was der richtige Ansatz für den Prozess der Promotion ist.
\bigskip

\noindent
Insbesondere sollen in dieser Arbeit
existierende Strategien für die Lösung des Problems
mit existierenden Tools erforscht werden.
Des Weiteren soll ein Prototyp einer neu vorgeschlagenen
Strategie entwickelt werden.
Im Anschluss soll der Prototyp im Rahmen eines Laborexperiments
mit bereits existierenden Strategien
verglichen werden.
\bigskip

% TODO: mein Lösungsvorschlag?


\noindent
Um das Ziel der Arbeit zu erreichen, wurden folgende Forschungsfragen identifiziert:

\begin{itemize}
	\item Wie kann die Promotion von Releases in GitOps-Umgebungen gestaltet werden?
	\begin{itemize}
		\item Welche Möglichkeiten bieten bestehende Tools für die Promotion von Releases mit mehreren Deployment-Umgebungen?
		\item Wie können Deployment-Umgebungen, sowie Promotion-Workflows abstrakt modelliert werden?
		\item Wie kann die abstrakte Modellierung verwendet werden, um eine standardisierte Lösung zur Promotion von Releases nach den GitOps-Prinzipien zu implementieren?
	\end{itemize}
\end{itemize}

