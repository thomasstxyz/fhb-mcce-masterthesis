\chapter{Schlussfolgerungen und Ausblick}

Obwohl die Adoption von GitOps einerseits einige Vorteile mit sich bringt,
werden andererseits neue Probleme erschaffen,
die zuvor noch nicht existierten,
oder bei zuvor gefolgten Strategien besser gelöst waren.
Außerdem sind mit GitOps die Ansprüche an
gute Software - wie so oft - gestiegen,
und es wurden imzugedessen neue Probleme identifiziert.
Diese neu identifizierten Probleme mögen sich für viele Organisationen derzeit lediglich
als Luxusprobleme herausstellen.
Jedoch können sich diese derzeitigen Luxusprobleme
mit steigendem Wachstum einer Organisation
und den damit einhergehenden steigenden Ansprüchen,
als reale und akute Probleme
entpuppen.
\bigskip

\noindent
Je höher die Komplexität und der benötigte Automatisierungsgrad einer Organisation, sowie dessen Adoption von DevOps und den Prinzipien wie CI/CD/CDP, 
desto wichtiger wird die vollständige Automatisierung der Promotion von Releases zwischen Umgebungen.

%Nachdem sich der entwickelte Prototyp als akzeptable Lösung für die Problemstellung zeigt,
%wäre es im nächsten Schritt unter anderem sinnvoll

%GitOps bringt Vorteile, aber auch Nachteile mit sich.
%Manche Punkte sind mit anderen bzw. früheren Ansätzen

% TODO: Ausblick, warum das wichtig ist...im Big Picture





