\chapter{Schlussfolgerungen und Ausblick}

Mit rasant steigender Nachfrage an Rechenzentren, welche zu 
einem großen Teil zum gesamten Stromverbrauch auf der Erde beitragen,
wird es immer wichtiger, auf die Energieeffizienz zu achten.
Energieeffizienz lässt sich an verschiedenen Punkten verbessern.
In dieser Arbeit soll die Effizienz von Workloads bzw. Applikationen in Kubernetes
dynamisch bestimmt werden.
Schließlich sollen mit den gewonnenen Metriken Workloads zu bestimmten Nodes zugeordnet werden können,
auf welchen sie am effizientesten laufen. 
Bestimmte Algorithmen funktionieren beispielsweise
effizienter oder weniger effizient auf verschiedenen Prozessor-Architekturen.
Das Auswählen der optimalen Node bzw. zugrundeliegenden Hardware
soll völlig transparent funktionieren.






