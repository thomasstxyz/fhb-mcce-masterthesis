\chapter{Conclusion and Future Work}

%In einer Conclusio wird keine neue Information präsentiert, sondern das bereits gesagte nochmal zusammenfassend wiedergegeben. Du möchtest hier dem Leser die wichtigsten Key Points deiner Arbeit mitgeben, so dass er sich ewig dran erinnernt.
%
%In welchem Forschungsfeld sind wir nochmal und was ist das eigentlich Problem dort?
%
%Welcher Lösungsvorschlag wurde präsentiert, um dieses Problem zu lösen und welche Methodik wird eingesetzt, um das zu schaffen?
%
%Ausblick...wenn wir das dann geschafft haben, was können wir dann mehr, was wir heute nicht können?
%
%Future Work -> welche neuen Forschungsfelder eröffnen sich dann, wo können wir ab dann weitermachen?

%\noindent
%The currently available GitOps tools
%do not provide an integrated solution to
%the problem of promoting releases between environments.
%This thesis aims at addressing the given problem.
%This is achieved by
%clearly defining the problem in distinct items,
%from which research objectives are then inferred.
%Following in designing and developing abstract models for
%environments and promotion processes;
%which are implemented in the produced artifact "\textit{GitOps Promotions Operator}".
%The artifact is demonstrated in a prototype serving as a proof-of-concept.
%Finally the research is evaluated by means of
%comparing the artifact's functionality with the solution objectives.
%\bigskip
%
%\noindent
%The developed "\textit{GitOps Promotions Operator}" prototype should serve as
%a modular extension to Kubernetes and existing GitOps tooling within the CNCF.
%It should provide a more streamlined and GitOps-native approach
%to the process of release promotion.
%For subsequent research projects the use of the prototype could
%be observed in a case study or tested in a field experiment,
%and adapted in another iteration of the applied methodology process model.
%The main goal for future work is the extensive adaption of the prototype
%to enable use in production by organizations or other projects.





The increasing adoption of a DevOps culture in organizations to develop applications and services quickly,
and reduce friction between people, communications and technical processes,
to ultimately decrease the time to market for new product releases,
has brought forward a new practice called GitOps.
%
One of the unresolved problems of the GitOps practice is
the process of promoting releases between multiple deployment environments.
Current tools in the ecosystem do not provide an integrated solution for this process.
Users are therefore inclined to build workflows which are constrained to specific
Git providers, GitOps engines,
workflow/pipeline systems, and configuration/templating tools. This can lead to tightly coupled setups,
and vendor lock-in.
%
In this research,
the given problem was addressed by designing uniform, standardised models for
defining GitOps-native deployment environments and promotion processes.
These models were implemented in a prototype as custom resources and controllers with the
operator pattern, as a Kubernetes extension.
This developed software artifact allows users to define abstract representations
of their environments, and how they want releases to be promoted between them.

The overall goal of the thesis was to provide a solution to the problem of
release promotion in GitOps environments.
%
In order to help with recognition and legitimization of the conducted research,
the methodology for conducting design science research in information systems
%\autocite{designScienceResearchMethodologyForInformationSystemsResearch}
was applied. The scientific method of semi-structured interviews was used
to support the problem identification and motivation.
A software artifact was designed and developed.
A prototypical implementation of the \textit{GitOps Promotions Operator}
was demonstrated, and its functionality was evaluated against the
research objectives.

Prior research on the concrete problem is focused on presenting
good practices and suggestions
which users need to manually implement themselves.
In addition, it is suggested to let external workflow/pipeline systems handle the promotion process,
or limit the amount of environments to one, in order to avoid having to do promotions altogether.
Conversely, this thesis brought forward
abstract models of environments and promotion processes,
which are implemented in the proposed prototype operator,
as Kubernetes custom resources and controllers, with the operator framework.
The prototype assesses the feasibility of
defining deployment environments and promotion processes declaratively,
following the GitOps principles.

The theoretical background on the topic was brought forward to the user,
in order to aid comprehension of the material within the thesis.
General definitions of terms,
fundamentals of DevOps and GitOps along related tooling and components were presented.
It was shown how GitOps changes the architecture and process of Continuous Deployment,
and how the promotion of releases is achieved without and with the GitOps approach.
The modeling approach of GitOps environments was discussed.
Emerging patterns like progressive delivery,
as well as the concept behind short-living environments were described.
The role of Kubernetes as a cloud native platform and its use cases beyond container orchestration were described.

\textit{Problem 1: Promotion is limited to container image},
relates to a frequent issue with currently available tooling in the GitOps ecosystem. 
Often times solely container image version tags are the focus with current tools when promoting
new versions or releases.
%It was discussed, that this is insufficient for some use cases.
Because it is sometimes required to handle all sorts of resources, not just the version tag of a container image,
an according research objective was defined.
%Especially when not using containerization technologies for runtime, this is an important problem to handle.
%In order to be able to provide a solution to this problem with a comprehensible approach,
%a solution objective was inferred and its requirements defined.
\textit{Objective 1: Arbitrary resources can be promoted},
defines a qualitative description of how the respective problem is supposed to be solved
by the developed artifact. The main idea is to offer the capability to promote arbitrary resources,
meaning any type of resource, instead of solely the container image version.
%The technical implementation in the proposed prototype foresees the functionality for
%promoting a list of filesystem paths inside the Git repository of the desired state to other environments.
%In addition these arbitrary resources should be able to be assigned a descriptive name,
%in order to identify the promotion subjects more easily.
%
\textit{Problem 2: Order of promotion to multiple environments},
states the fact, that it is not a straight-forward process of how the order of promotion
through multiple GitOps environments can be setup.
%When adhering to the principles of GitOps
%and sticking with the asynchronous deployment process (described in the
%\nameref{theoretical-background} chapter)
%there is no streamlined approach or tooling, that automates this, while still sticking to the asynchronous process.
\textit{Objective 2: Strict flow of promotion through environments},
defines the requirements for the proposed prototype,
in regards to the according problem of having a certain order of promotion through environments or stages.
The objective describes the capability for defining a certain order of environments, in which releases traverse through.
In addition, this solution objective opens up the possibility to setup promotion in stages, in which
certain environments must be deployed to first, before the release can deploy to other specified environments.
%
\textit{Problem 3: Dependencies can not be defined},
relates to the problem that when wanting to promote a new release from one environment to another environment,
it is not easily achievable with the available tools to specify certain dependencies, like other workloads or
microservices in the same or another environment, or dependencies from external sources.
This is especially desirable for evaluating test results or other metrics, before triggering the promotion.
\textit{Objective 3: Dependencies of a promotion},
describes how the respective problem of being able to specify dependencies for a promotion,
could be solved in the proposed prototype. While the minimum dependency is the successful deployment
of the workload of a release,
it may also be desirable to specify other resources or workloads which need to be in a certain state,
before triggering a promotion.
%
\textit{Problem 4: Provider and tool dependency},
draws attention to the common problem of being dependent on particular tools and providers.
The more complex the Continuous Delivery is setup for a project,
the more difficult it is to de-couple or switch providers for certain components.
%Furthermore, since many tools in the GitOps ecosystem are not very mature in their development and adoption,
%it is of use that components are loosely coupled and can be exchanged with alternatives in the future.
\textit{Objective 4: Vendor-neutral, tool-agnostic},
defines the requirements of how a vendor-neutral and tool-agnostic prototype can be implemented.
The promotions operator supports any GitOps engine, Git provider, and configuration tool.
%The main components which are desirable to support all alternatives,
%for being able to switch between them,
%are the Git providers (e.g. GitHub, GitLab),
%the GitOps engines (e.g. Argo, Flux),
%the configuration/templating tools (e.g. Kustomize, Helm).
%
Additionally,
related ideas and approaches were discussed by the interview partners.
These points were not directly considered for the conducted design science in the prototype,
however they were discussed later in the thesis.

The proposed prototype was presented.
The asynchronous nature of GitOps deployments, and where
the operator prototype fits within this architecture was described.
Abstract models for the environment and promotion custom resources as well as their prototype
design as declarative Kubernetes custom resources was described.
The implementation of these custom resources was shown in the form of mockups
of Kubernetes custom resources in the Yaml format.
Alternative mockup designs were shown as a way to draw attention to the fact
that the actual design of the API specification is not cast in stone.
Moreover, the API specification should be tested with users,
and should be adapted for usability and ease of use.
The translation of the API specification into Go types was described,
and finally the implemented controller logic of both the environment, as well as
the promotion controller was presented.
The developed artifact of the prototype operator was
demonstrated in the context of a proof-of-concept use case.
The demonstration of the prototype's functionality was then evaluated against
%the research objectives defined in
%\ref{interviews:definitionSolutionObjectives} \nameref{interviews:definitionSolutionObjectives}.
the research objectives.
%
The results of the conducted research were presented, primarily by means of
presenting and describing the designed and developed operator prototype, and the learning
from the prototyping process.
In addition, the interviewed working professionals gave several interesting insights into the problem
statement from their point of view.
The results of the research were evaluated on how they provide a solution to the research problem.
The implemented functionality of the prototype was evaluated against the research objectives.
This was done by comparing the qualitative descriptions of the objectives with the actual observed results
in the demonstration of the prototype in the proof of concept.
% ?? feedback for the prototype from the working professionals??
For the research question RQ 1.1,
a possible solution was presented by means of
describing the design of a prototype of a Kubernetes operator for handling GitOps promotions.
For research question RQ 1.2,
a possible implementation of the abstract models was presented, in the form of
declarative custom resources, which extend the Kubernetes API.
The overarching research question 1 is the combination of the sub research questions.
The thesis proposes one possible way of how the research problem can be addressed,
namely the promotion of releases in GitOps environments can be designed.
This concrete research does not try to propose a definitive answer or solution to the research question.
%
The results, learnings, and evaluations of the research were discussed and interpreted.
The meanings behind the specific results are brought forward in more detail.
Moreover, interpretations and implications of the results and evaluations were presented.
Learnings from implementing the prototype were presented, namely
ideas about the user experience, security considerations, the use at scale, abstractions and modularity.
Alternative approaches for promoting releases were presented.

Further suggestions and point of references for future research on this topic
and the developed prototype were presented.
These included
further research and development of the proposed prototype,
which is about testing its user experience, evaluating integration with
other GitOps tools. Generally the aim is to enhance the prototype
to make it mature for production use.
The idea of rolling production environments by interview partner 1 was discussed.
It describes a somewhat different approach for promotions in GitOps,
where less environments are needed, but for each new release the 
production environment is re-created with the new versions, and
progressive delivery is done not only on an application level, but on the whole
infrastructure stack together with the end user application or service on top.
This is to further improve immutability and versioning to increase resiliency.
The idea of the problem with the overview of GitOps repositories by interview partner 2
was discussed.
It is about the somewhat missing feature of a quick and easy to understand overview over a bare GitOps repository.
Depending on the used configuration/templating tool, a setup looks different.
Deployment environments can be represented, however it is not possible for the user
to know what the target environment is, or where the GitOps definition is deployed to in general.
Moreover it was discussed that it would make sense for future work
to research a wide range and variety of organizations and do a survey on
their requirements, their issues and how they imagine a solution.
An initiative towards standardized GitOps promotions should be made,
because for open-source tooling the aim should be to strive for functionality
that can be used by everyone,
instead of providing tailored tooling which may only be beneficial for specific use cases
and organizations.




















%Although the adoption of GitOps brings some advantages on the one hand,
%on the other hand, it creates new problems
%that did not exist before,
%or were better solved with previously followed strategies.
%In addition, with GitOps the demands of good software have increased
%- as so often -
%and new problems have been identified as a result.
%These newly identified problems may currently turn out 
%to be merely
%luxury problems for many organizations.
%However, as an organization grows, these current luxury problems
%with the increasing growth of an organization
%and the increasing demands that come with it,
%may emerge as pressing problems.
%\bigskip
%
%\noindent
%The higher the complexity and the needed degree of automation
%of an organization,
%as well as the adoption of DevOps
%and the principles like
%Continuous Integration (CI),
%Continuous Delivery (CD),
%Continuous Deployment (CDP) -
%the more important it becomes to fully automate the promotion of releases in GitOps environments.
%
%%Nachdem sich der entwickelte Prototyp als akzeptable Lösung für die Problemstellung zeigt,
%%wäre es im nächsten Schritt unter anderem sinnvoll
%
%%GitOps bringt Vorteile, aber auch Nachteile mit sich.
%%Manche Punkte sind mit anderen bzw. früheren Ansätzen
%
%% Ausblick, warum das wichtig ist...im Big Picture





