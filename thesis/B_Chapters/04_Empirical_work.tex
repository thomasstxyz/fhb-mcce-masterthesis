%\chapter{Empirical Work}


\chapter{Interviews with Working Professionals}

\section{Problem Identification \& Motivation}
\section{Definition of Solution Objectives}
\section{Summary}



\chapter{GitOps Promotions Operator Prototype}

This chapter describes the developed prototype,
called the GitOps Promotions Operator.

\section{Design \& Development}

The Design of the GitOps Promotions Operator
is based on the underlying constraints of the Kubebuilder framework,
discussed in section \ref{theoretical-background:kubernetes} of this thesis.
The main constraints the framework comes with, are the
custom resource definition + controller pattern.

\begin{figure}[h]
	\centering
	\includegraphics[width=1.00\linewidth]{assets/crd-and-controller.png}
	\caption{Custom Resource Definition and Controller.
		%		(\citeauthor{ref}, \citeyear{ref}).
	}
	\label{fig:crd-and-controller}	
\end{figure}

As the initial step of the design phase,
abstract model definitions are designed.
Their key properties are identified.
Then a sample mockup is designed as
a custom resource definition in Yaml format.
Afterwards this idea of the model definition is 
translated into custom struct types of the
Go programming language,
which is chosen by the Kubebuilder framework.

\subsection{Abstract Models}
	
The first step of the initial design phase is the creation of abstract models.
In order to be able to represent environments and promotions,
the requirement is to at least start with two abstract models for each 
the environment and promotion.

The Environment represents a GitOps environment,
which is a Git repository + path.
The Git repository can be a clone URL to the repository,
and the path is the relative filesystem path, which points to the 
environment, inside the repository.

\begin{figure}[h]
	\centering
	\includegraphics[width=1.00\linewidth]{assets/gitops-env-repo-and-path.png}
	\caption{GitOps Environment.
		%		(\citeauthor{ref}, \citeyear{ref}).
	}
	\label{fig:gitops-env-repo-and-path}	
\end{figure}

The abstract model for a GitOps environment needs at least the following properties:

\begin{itemize}
	\item URL of the source Git repository
	\item path pointing to the environment inside the repository
\end{itemize}

The URL has the format of a HTTP(S) or SSH URL,
which links to the Git repository,
e.g.:

\begin{itemize}
	\item \lstinline|http://localhost:8080/org/repo|
	\item \lstinline|https://gitprovider.com/org/repo|
	\item \lstinline|ssh://git@gitprovider.com:org/repo|
\end{itemize}

The path has the format of a typical unix style filesystem path.
It starts relative from the root of the given Git repository,
and points to the directory, which represents the GitOps environment.
Examples for a path are the following:

\begin{itemize}
	\item \lstinline|path/to/env|
	\item \lstinline|/path/to/env|
	\item \lstinline|./path/to/env|
	\item \lstinline|./path/to/env/|
\end{itemize}

Note, that these example paths all represent the same directory,
these are just alternative notations.

The abstract model for a GitOps promotion needs at least the following properties:

\begin{itemize}
	\item source environment
	\item target environment
	\item promotion subjects
	\item promotion strategy
\end{itemize}

The source environment defines the environment resource,
where a promotion subject is promoted from.
The target environment defines the environment resource,
where a promotion should promote to.

A promotion subject can be potentially many different things.
In the case of this prototype,
a promotion subject is a file or directory,
which is copied from the source to the target environment.
Examples of such files or directories are the following:

\begin{itemize}
	\item \lstinline|kustomization.yaml|
	\item \lstinline|./component/cert-manager/kustomization.yaml|
	\item \lstinline|./helm-values-prod.yaml|
\end{itemize}

Note, that the relative paths of the promotion subjects,
are relative to the paths of the environment, defined earlier.
An example of the \lstinline|kustomization.yaml|, would look like this:

\lstinline|./path/to/env/kustomization.yaml|

As an example - but outside the scope of the current prototype -
a promotion subject could also be fetched from another source,
like an artifact registry, or be any other type of data
and updated/promoted in the target environment,
e.g. a version tag, helm values, etc.

With the current prototype,
the promotion strategy is to raise a pull request at the Git provider,
with the changes proposed by the promotion.
A human can then review the changes and optionally approve and merge the pull request.
After merging, the promotion will have taken place.

Alternatively to a pull request, the changes could be directly
commited and pushed to the target environment,
without human interaction. This strategy should require different means
of automated or otherwise external or additional checks, in order to ensure a safe promotion.

Now that the abstract models are designed,
they need to be implemented.
Since the decision for the prototype is to be developed
with the Kubebuilder framework and follow its style,
mockups of the custom resources in Yaml format
will be created as a next step.





\subsection{Mockups of Custom Resources}

%to write mockups
%of custom resources in YAML format,
%since this is what the end user will interface with when the application is finished.
%First and foremost the handling and user experience has to be seamless and make sense
%for the user; also it shouldn't take any extra miles for the user to take,
%just to get started.
%The YAML representation of the custom resource should be similar to other ones
%from the Kubernetes core types.

When the abstract models have been specified,
they can be actually implemented with the framework
as Kubernetes custom resource definitions.

Users will mainly be dealing with the custom resources in a Yaml format.
Yaml keys should be intuitive and make sense to the user.
It also helps if they follow the core Kubernetes definitions regarding naming conventions.
An example for a naming convention is the "Ref" suffix for yaml keys.
This suffix is typically appended to keys which represent a reference to another Kubernetes
object.
For example, "secretRef" says that this field refers to a Kubernetes secret resource.

A possible mockup for a GitOps environment -
that is a first prototype implementation of the abstract model into a custom resource -
could look like the following.

\lstinputlisting{assets/files/environment-mockup.yaml}

In this mockup the git reference branch main is also specified \\
in the \lstinline|.spec.source.ref.branch| field.

A possible mockup for a GitOps promotion 
could look like the following.

\lstinputlisting{assets/files/promotion-mockup.yaml}

In the promotion mockup definition,\\
there are four main fields within the \lstinline|.spec|.
These represent the minimum properties of the previously defined abstract definition.
It is to note, that
the \lstinline|.spec.copy| field represents the promotion subjects.
It is a list of items, where each item contains
a \lstinline|name|, \lstinline|source| and \lstinline|target|.
The \lstinline|name| defines a custom name.
The \lstinline|source| and \lstinline|target| fields together define a
file copy operation,
where the \lstinline|source| is the relative path from the source environment,
and the \lstinline|target| is a relative path from the target environment.

\subsection{Alternative Mockups}

The following alternative mockups,
for the custom resource definitions,
i.e. the design of the declarative API,
are suggested, but now implemented in the current prototype.

\subsubsection*{Promotion Subjects defined in each Environment resource}

Alternatively, the promotion subjects could also be specified
in the environment resource.
Then the environment could like the following:

\lstinputlisting{assets/files/environment-mockup-alt-1.yaml}

In the promotion definition,
it would then suffice to specify
a list of promotion subjects.

\lstinputlisting{assets/files/promotion-mockup-alt-1.yaml}

This alternative design allows that each environment could have
a unique path of a specific promotion subject defined.
Now if a promotion is spanning over multiple environments,
they could each specify their own unique path to a promotion subject.
The promotion subject is declared in the promotion,
but the actual path is defined per each environment.

\subsection{Translation to Go types}

Once the mockups of the custom resources in Yaml format are done,
the declarative structure can be translated to custom Go types.

The specification of the Environment resource results in the following code:

\lstinputlisting{assets/files/environmentSpec-type.go}

The type EnvironmentSpec represents the \lstinline|.spec| Yaml field.

What also needs to be defined is the status subresource.
In the status fields, the controller can save the current/actual state
of the resource during runtime.
While \lstinline|.spec| defines the desired state,
\lstinline|.status| defines the actual state, as observed by the controller.

% TODO: maybe show example of .status yaml

\lstinputlisting{assets/files/environmentStatusSpec-type.go}

The specification of the Promotion resource results in the following code:

\lstinputlisting{assets/files/promotionSpec-type.go}

For the promotion,
a status subresource is also defined.
In the status - the actual state of the resource as observed by the controller -
most importantly the metadata of the currently opened pull request is saved,
which the controller will pick up on every consecutive reconciliation
of the same promotion object.

\lstinputlisting{assets/files/promotionStatusSpec-type.go}

The full source code of the types can be found in Appendix
\ref{appendix:source-code}.

Once the Go types are defined,
the controller logic can be written.

\subsection{Controller Logic}

For the environment API,
a controller is written.
For this prototype the following logic was implemented.
First, the source git repository is cloned and checked out,
with the appropriate authentication options, if it is private.
If it succeeds, the Ready condition is set in the status subresource,
which will mark it available and ready for the promotion.

\begin{enumerate}
	\item test the clone of Git repository with authentication
	\item checkout reference branch locally
	\item mark environment as ready
\end{enumerate}

For the promotion API,
the following controller logic is implemented.
First, the controller checks if the source and target environments are ready,
if they are not yet ready, the controller cancels the reconciliation immediately.
Then the source and target environments are cloned.
Next the controller checks if there is a pending/open pull request,
this information is retrieved from the object's status, and then checked
if still up to date via the Git provider's API.
Afterwards the controller executes the promotion tasks,
which are the copy operations with the current state of the prototype.
Now if there were changes since the last reconciliation, the new commits
are pushed to the pull request branch.
Lastly, a pull request will be raised, if not yet done during a previous reconciliation.

\begin{enumerate}
	\item ensure that the source and target environments are ready
	\item clone source and target repositories
	\item check for a pending promotion (open Pull Request)
	\item execute the promotion copy operations
	\item push new commits to PR branch, if there were differences between source \& target environments
	\item create new PR, if not yet opened
\end{enumerate}

The full source code of the controllers can be found in Appendix
\ref{appendix:source-code}.






\section{Proof Of Concept Demonstration}

The following section demonstrates the in-context usage of the
developed prototype - the GitOps Promotions Operator - in a proof of concept.
The use case described as follows is created for the purpose of this demonstration.

This use case deals with a setup with multiple deployment environments.
There are two non-critical environments \lstinline|dev| and \lstinline|qa|,
and two production environments \lstinline|prod-1| and \lstinline|prod-2|.
The GitOps definitions of the non-critical environments are living inside the same
Git repository \lstinline|mtpoc-infra-1|,
and each production environment lives in its own separate Git repository
\lstinline|mtpoc-infra-2| for \lstinline|prod-1|,
and \lstinline|mtpoc-infra-3| for \lstinline|prod-2|.
In general, the application version shall be promoted with a strict flow
through the environments, one after the other.
An overview of the given setup can be seen in the table \ref{table:poc-environments-setup}.

\begin{table}[h]
\begin{center}
	\begin{tabular}{||c c c||} 
		\hline
		Order & Environment & Source Repository \\ [0.5ex] 
		\hline\hline
		1 & dev & mtpoc-infra-1 \\ 
		\hline
		2 & qa & mtpoc-infra-1 \\
		\hline
		3 & prod-1 & mtpoc-infra-2 \\
		\hline
		4 & prod-2 & mtpoc-infra-3 \\ [1ex]
		\hline
	\end{tabular}
	\caption{PoC Environments Setup}
	\label{table:poc-environments-setup}
\end{center}
\end{table}

The GitOps environment is centered around the used configuration management tool
Kustomize, and generally structured for all environments as below:

\begin{lstlisting}
.
|-- app-version
|   `-- kustomization.yaml
|-- kustomization.yaml
`-- settings
    `-- deployment.yaml
\end{lstlisting}

This structure adheres to the constraints of the currently available
copy operation promotion type, which can copy files and directories.
This means the configuration components which need to be promoted,
should be defined in separate files or directories.
This is needed, in order to only promote e.g. the application image version,
while leaving other configuration untouched, and specific to an environment.
With the Kustomize configuration tool, it is possible to split
parts of the main \lstinline|kustomization.yaml| into other separated files,
with the components feature.

In this use case, the value of the application's image version lives within the 
app-version component. This is configured in the main \lstinline|kustomization.yaml|
like this:

\begin{lstlisting}
components:
- app-version
\end{lstlisting}

The \lstinline|./app-version| directory contains a \lstinline|kustomization.yaml| file,
with typical Kustomization specification.
In this case, the images feature of Kustomize is used for configuring the application's
image version tag.

\begin{lstlisting}
apiVersion: kustomize.config.k8s.io/v1alpha1
kind: Component
images:
- name: ghcr.io/stefanprodan/podinfo
newTag: 6.3.4
\end{lstlisting}

Now the goal is to configure a promotion for the app-version component.
To achieve this, 
first, a \lstinline|Environment| resource needs to be created
for all environments respectively.
Only the \lstinline|dev| environment is shown here,
the other three environment definitions follow the same schema,
but are omitted for the sake of brevity.

\lstinputlisting{assets/files/dev-environment.yaml}

%\lstinputlisting{assets/files/qa-environment.yaml}
%
%\lstinputlisting{assets/files/prod-1-environment.yaml}
%
%\lstinputlisting{assets/files/prod-2-environment.yaml}

Now the specified secrets must be created.
The API token is required for the creation of pull requests by the promotion controller;
it must be stored in a Kubernetes generic secret resource in a key named \lstinline|token|,
and can be created with the following command:

\lstinline|kubectl create secret generic github-api-token --from-literal=token="gh..."|

With the current prototype, also a secret for the SSH connection to push and pull
the repository, needs to be created.
For this, a ssh key pair needs to be created by the user. Its public key needs to 
be set as a deploy key at the Git provider,
and its private key needs to be stored in a
Kubernetes generic secret resource in a key named \lstinline|private|.

\lstinline|kubectl create secret generic github-api-token --from-literal=private="--..."|

When all the four environments are created, the \lstinline|Promotion| resources
can be defined.
In this use case, three promotion resources are needed for the ability to
promote between all four environments with a straight flow - promoting from one to the next.
Only the \lstinline|dev-to-qa| promotion is shown here,
the other two definitions follow the same schema,
but are omitted here for the sake of brevity.

\lstinputlisting{assets/files/dev-to-qa.yaml}

%\lstinputlisting{assets/files/qa-to-prod-1.yaml}
%
%\lstinputlisting{assets/files/prod-1-to-prod-2.yaml}





At this point, all which is needed for promoting is configured.
When the status of all the environment resources involved in a promotion,
have a ready status condition,
a promotion will trigger.

\lstinputlisting{assets/files/env-dev-status.yaml}

At this point, the controller logs can be observed.

\begin{lstlisting}
2023-04-16T12:10:40Z    INFO    Begin reconciling Promotion     {"controller": "promotion", "controllerGroup": "promotions.gitopsprom.io", "controllerKind": "Promotion", "Promotion": {"name":"dev-to-qa","namespace":"default"}, "namespace": "default", "name": "dev-to-qa", "reconcileID": "ba3775ae-d8b1-40de-ad7b-4f42f8534686", "name": {"namespace": "default", "name": "dev-to-qa"}}
2023-04-16T12:10:46Z    INFO    Created new pull request        {"controller": "promotion", "controllerGroup": "promotions.gitopsprom.io", "controllerKind": "Promotion", "Promotion": {"name":"dev-to-qa","namespace":"default"}, "namespace": "default", "name": "dev-to-qa", "reconcileID": "ba3775ae-d8b1-40de-ad7b-4f42f8534686", "WebURL": "https://github.com/thomasstxyz/mtpoc-infra-1/pull/1"}
2023-04-16T12:10:46Z    INFO    Reconciled Promotion successfully       {"controller": "promotion", "controllerGroup": "promotions.gitopsprom.io", "controllerKind": "Promotion", "Promotion": {"name":"dev-to-qa","namespace":"default"}, "namespace": "default", "name": "dev-to-qa", "reconcileID": "ba3775ae-d8b1-40de-ad7b-4f42f8534686", "duration": "6.050374419s", "nextReconcile": "300s"}
\end{lstlisting}

A new pull request
%with the number \lstinline|1|
%at the Web URL \\
%\url{https://github.com/thomasstxyz/mtpoc-infra-1/pull/1} \\
has been created by the controller.

The promotion's status will also reflect, that
a pull request is open for review.

\lstinputlisting{assets/files/prom-dev-to-qa-status.yaml}

The open pull request is ready for review in the Git provider's web interface.
%\begin{figure}[h]
%	\centering
%	\includegraphics[width=1.00\linewidth]{assets/new-promotion-github.png}
%	\caption{New Git Pull Request.
%		%		(\citeauthor{ref}, \citeyear{ref}).
%	}
%	\label{fig:new-promotion-github}	
%\end{figure}
\begin{figure}[h]
	\centering
	\includegraphics[width=1.00\linewidth]{assets/prom-pr-dev-to-qa.png}
	\caption{Pull Request for Promotion from dev to qa.
		%		(\citeauthor{ref}, \citeyear{ref}).
	}
	\label{fig:prom-pr-dev-to-qa}	
\end{figure}

The changed difference introduced by the commit can be viewed:

\begin{lstlisting}
- newTag: 6.3.3
+ newTag: 6.3.4
\end{lstlisting}

The \lstinline|dev-to-qa| promotion requested the change of the image version from
\lstinline|6.3.3| to \lstinline|6.3.4|.

Now, since the \lstinline|qa| and the \lstinline|prod-1| environments
also differ,
a pull request has also been created for this promotion.

%\begin{figure}[h]
%	\centering
%	\includegraphics[width=1.00\linewidth]{assets/prom-pr-qa-to-prod-1.png}
%	\caption{Pull Request for Promotion from qa to prod-1.
%		%		(\citeauthor{ref}, \citeyear{ref}).
%	}
%	\label{fig:prom-pr-qa-to-prod-1}	
%\end{figure}

The \lstinline|qa-to-prod-1| promotion requested the change of the image version from
\lstinline|6.3.2| to \lstinline|6.3.3|.

\begin{lstlisting}
- newTag: 6.3.2
+ newTag: 6.3.3
\end{lstlisting}

Now, since the \lstinline|prod-1| and the \lstinline|prod-2| environments
also differ,
a pull request has also been created for this promotion.

%\begin{figure}[h]
%	\centering
%	\includegraphics[width=1.00\linewidth]{assets/prom-pr-prod-1-to-prod-2.png}
%	\caption{Pull Request for Promotion from prod-1 to prod-2.
%		%		(\citeauthor{ref}, \citeyear{ref}).
%	}
%	\label{fig:prom-pr-prod-1-to-prod-2}	
%\end{figure}

The \lstinline|prod-1-to-prod-2| promotion requested the change of the image version from
\lstinline|6.3.1| to \lstinline|6.3.2|.

\begin{lstlisting}
- newTag: 6.3.1
+ newTag: 6.3.2
\end{lstlisting}

If the \lstinline|dev| environment advances the application image version further,
the pull request for the \lstinline|dev-to-qa| will be updated with another commit.
Note that the previous commit is not overwritten, 
but the commit history is kept on the pull request branch - now there are two commits on the branch.

\begin{figure}[h]
	\centering
	\includegraphics[width=1.00\linewidth]{assets/prom-pr-dev-to-qa-round2.png}
	\caption{Pull Request updated for Promotion from dev to qa.
		%		(\citeauthor{ref}, \citeyear{ref}).
	}
	\label{fig:prom-pr-dev-to-qa-round2}	
\end{figure}

The difference for the \lstinline|dev-to-qa| promotion is now:

\begin{lstlisting}
- newTag: 6.3.3
+ newTag: 6.3.5
\end{lstlisting}

%\begin{figure}[h]
%	\centering
%	\includegraphics[width=1.00\linewidth]{assets/prom-pr-dev-to-qa-round2-diff.png}
%	\caption{Pull Request updated for Promotion from dev to qa (diff).
%		%		(\citeauthor{ref}, \citeyear{ref}).
%	}
%	\label{fig:prom-pr-dev-to-qa-round2-diff}	
%\end{figure}

Now if the pull request for the \lstinline|dev-to-qa| promotion
is approved and merged by a human,
the promotion will actually take effect.
This will results in the \lstinline|qa-to-prod-1| promotion pull request being updated
by the promotion controller.
Version \lstinline|6.3.5| is now requested for promotion to the
\lstinline|prod-1| environment.

The difference for the \lstinline|qa-to-prod-1| promotion is now:

\begin{lstlisting}
- newTag: 6.3.2
+ newTag: 6.3.5
\end{lstlisting}

Once reviewed, approved and merged by a human,
the promotion of version \lstinline|6.3.5| to the \lstinline|prod-1| environment
will result in further propagation of version \lstinline|6.3.5|
to the \lstinline|prod-2| environment.

The difference for the \lstinline|prod-1-to-prod-2| promotion pull request is now:

\begin{lstlisting}
- newTag: 6.3.1
+ newTag: 6.3.5
\end{lstlisting}

Once reviewed, approved and merged by a human,
the promotion of version \lstinline|6.3.5| to the \lstinline|prod-2| environment
will take effect.

The described demonstration showed,
how an application version can be promoted across multiple environments,
while ensuring a strict flow of promotion of \\
e.g. \lstinline|dev --> qa --> prod-1 --> prod-2|




%For this demonstration, it is supposed that
%the latest version \lstinline|6.3.5| is bad and should not be promoted,
%instead the \lstinline|dev| environment shall be rolled back to \lstinline|6.3.4|.






















\section{Evaluation Of Proof-of-Concept Use Cases}











\section{Summary}







%\chapter{Interviews with Working Professionals}
%
%\section{Categorisation of Findings}
%\section{Common Problem Definitions}
%\section{...?}
%
%
%\chapter{Definition of Solution Objectives}
%
%\section{People \& Communication Perspective}
%\section{Technical Perspective}
%\section{...?}
%
%
%\chapter{Prototype Design and Development}
%
%\section{Architecture}
%\section{Functionality}
%\section{...?}
%
%\chapter{Proof-of-Concept Demonstration}
%
%\section{Setup and Use with Kustomize}
%\section{Setup and Use with Helm}
%\section{Multiple Environments in same Stage}
%\section{Scalability}
%\section{...?}















%
%\section{Instruction included in the original FHBgld word processor template}
%Die Durchführung der empirischen Untersuchung ist nachvollziehbar zu dokumentieren sowie auch die dabei aufgetretenen Probleme und deren Behandlung. Der Umfang ergibt sich aus der Art der Bearbeitung.  
%
%Tabelle 1 zeigt ein Bespiel für eine Tabelle. 
%
%\begin{figure}[ht]
%	\centering
%	\includegraphics[width=0.7\linewidth]{figures/Word_Table}
%	\caption{Screenshot example from FHBgld word processor template}
%	\label{fig:wordtable}
%\end{figure}
%Abbildung 1 zeigt ein Beispiel für eine Abbildung oder Grafik.
%\begin{figure}
%	\centering
%	\includegraphics[width=0.7\linewidth]{figures/Word_Diagram}
%	\caption{Screenshot example from FHBgld word processor template}
%	\label{fig:worddiagram}
%\end{figure}
%\linebreak
%Mathematisch werden die Zusammenhänge wie im Figure \ref{fig:wordformel} beschrieben.
%\begin{figure}
%	\centering
%	\includegraphics[width=0.7\linewidth]{figures/Word_Formel}
%	\caption{Screenshot example from FHBgld word processor template}
%	\label{fig:wordformel}
%\end{figure}
%
%\section{Tables and Images with \LaTeX}
%One of the great advantages of \LaTeX{} is that all it needs to know is
%the structure of a document, and then it will take care of the layout
%and presentation itself.  So, here we shall begin looking at how exactly
%you tell \LaTeX{} what it needs to know about your document.
%
%\subsection{Tables}
%In this sub-section, a simple table is inserted. To add reference to the table, see (cf. Table~\hyperref[tab:tableexample0]{\ref{tab:tableexample0}}):
%
%%A simple table.  The center environment is first set up, otherwise the
%%table is left aligned.  The tabular environment is what tells Latex
%%that the data within is data for the table.
%% https://en.wikibooks.org/wiki/LaTeX/Tables
%\begin{table}[htb]
%	\begin{tabular}{|b{7cm}|c|}
%		%The tabular environment is what tells Latex that the data within is
%		%data for the table.  The arguments say that there will be two
%		%columns, both left justified (indicated by the 'l', you could also
%		%have 'c' or 'r'.  The bars '|' indicate vertical lines throughout
%		%the table.
%		
%		\hline  % Print horizontal line
%		\fontsize{11pt}{12pt}\selectfont Command & Level \\ \hline  % Columns are delimited by '&'.  And
%		%rows are delimited by '\\'
%		\fontsize{10pt}{14pt}\selectfont Some sections to provide some examples: & \\
%		\texttt{\textbackslash part\{\emph{part}\}} & -1 \\
%		\texttt{\textbackslash chapter\{\emph{chapter}\}} & 0 \\
%		\texttt{\textbackslash section\{\emph{section}\}} & 1 \\
%		\texttt{\textbackslash subsection\{\emph{subsection}\}} & 2 \\
%		\texttt{\textbackslash subsubsection\{\emph{subsubsection}\}} & 3 \\
%		\texttt{\textbackslash paragraph\{\emph{paragraph}\}} & 4 \\
%		\texttt{\textbackslash subparagraph\{\emph{subparagraph}\}} & 5 \\
%		\hline
%		
%	\end{tabular}
%	\caption{some description of the table}
%	\label{tab:tableexample0}
%\end{table}
%
%\subsubsection{More tabular examples}
%
%First, a plain simple example for a FHBgld table, see table~\hyperref[tab:tab:tableexample1]{\ref{tab:tableexample1}}.
%
%\begin{table}[h]
%	\centering
%	\begin{tabular}{|b{1cm}|b{2cm}|b{3cm}|b{4cm}|}
%		\hline
%		\multicolumn{4}{|l|}{\fontsize{11pt}{12pt}\selectfont\noindent First line in 11pt fontsize } \\ \hline
%		1cm & 2cm & 3cm & 4cm \\ \hline
%		from & here on & the table & font size \\ \hline
%		will & be as & defined & in class, that is 10pt\footnote{yes, really!} \\ \hline
%		will & be as & defined & in class, that is 10pt\footnote{yes, really!} \\ \hline
%		will & be as & defined & in class, that is 10pt\footnote{yes, really!} \\ \hline
%		will & be as & defined & in class, that is 10pt\footnote{yes, really!} \\ \hline
%		will & be as & defined & in class, that is 10pt\footnote{yes, really!} \\ \hline
%	\end{tabular}
%	\caption{some description of the table}
%\label{tab:tableexample1}
%\end{table}
%
%Next, a table with nine columns, see table~\hyperref[tab:tableexample2]{\ref{tab:tableexample2}}.
%
%\begin{table}[h]
%	\centering
%	\begin{tabular}{|*{9}{l|}}
%		\hline
%		{\fontsize{11pt}{12pt}\selectfont This} & {\fontsize{11pt}{12pt}\selectfont table} & {\fontsize{11pt}{12pt}\selectfont has} & {\fontsize{11pt}{12pt}\selectfont way} & {\fontsize{11pt}{12pt}\selectfont too } & {\fontsize{11pt}{12pt}\selectfont many} & {\fontsize{11pt}{12pt}\selectfont columns}, & {\fontsize{11pt}{12pt}\selectfont does'nt} & {\fontsize{11pt}{12pt}\selectfont it?} \\ \hline
%		One & Two & Three & Four & Five & Six & Seven & Eight & Nine! \\ \hline
%		At & least & the & first & column & has & 11pt & font & size. \\ \hline
%	\end{tabular}
%	\caption{some description of the table}
%	\label{tab:tableexample2}
%\end{table}
%
%\subsection{Images}
%% Here is how to insert an image as a figure. There is a lot more you can do
%% when inserting images, check out: https://en.wikibooks.org/wiki/LaTeX/Importing_Graphics
%
%\begin{figure}[h]
%	\centering
%	\includegraphics[width=0.3\textwidth]{figures/logo_nontransparent.jpg}
%	\caption{Image Example}
%	\label{fig:image_example}
%\end{figure}
%
%When an image is inserted, you can refer to it like this (cf. Figure~\hyperref[fig:image_example]{\ref{fig:image_example}}).
%
%\subsubsection{A Subsubsection}
%As one last example, this is how you can insert a sub-sub-section! Have fun
%writing your thesis with \LaTeX{}!
%
%\lipsum[2-3]
%\raggedbottom
%
%\pagebreak
