\chapter{Future Work}
\label{future-work}

In this chapter,
further suggestions for future research on this topic and the developed prototype are presented.

In addition,
some interesting ideas that came out of this research are discussed,
and how they could be further researched in future work either by the researcher of this thesis or other researchers.



\section*{Testing the User Experience of the Prototype}

The proposed software prototype should be evaluated against its user experience,
in order to find out if users have difficulty doing the initial setup,
or any other troubles.
The API, the custom resources, which the operator provides can be changed to fit the users needs.
Additional functionality which is required by users, can be added.
Since the operator works with any other GitOps tooling the users bring from their individual GitOps setup,
it can lack a bit of ease of use. The integration into other GitOps tooling like Argo or Flux
should be evaluated, and examined if such an integration is worth it, in order to provide better convenience.

\section*{Rolling Production Environments}

The term rolling production environments was coined by interview partner 1 during the interview.
It was discussed in section \ref{insights-related-ideas-approaches} earlier in this thesis.
The term represents a future idea of where promotions and rollouts of new releases could be heading to.
This approach steers in a somewhat other direction, than what this thesis researched.
Instead of having fixed testing environments or stages, which a new release needs to pass,
before being rolled out to the production environment,
this approach conversely foresees only the single environment,
which is actually used for production.

This approach builds upon the progressive delivery pattern,
which can be implemented with tools like Flagger or Argo Rollouts.
Instead of just doing progressive delivery of an application (i.e. Kubernetes deployment / container image),
the progressive delivery would be done on a whole environment (i.e. Kubernetes cluster).
This has the advantage of further limiting the chance of supporting services and applications
influencing the actual user-facing application in a bad way, i.e. breaking the service, making it unavailable for its users.
Each different supporting application constantly changes versions, and all these changes could potentially break some other dependency.
Some infrastructure components are sometimes even updated without a version history,
eliminating the possibility to roll back to a safe and working state.

One possible future work that could be done for evaluating this new approach,
could be to evaluate the costs and feasibility of this approach.
According to interview partner 1, tools like the Weave GitOps Terraform Controller
have made it a lot easier to achieve the idea of rolling production environments.
However since a whole production environment would be dynamically created with each new release,
this could introduce higher costs.
Especially in a cloud environment, where billing is done on-demand per minute used,
and where the costs can be an important factor for deciding between different architectures.

\section*{Overview of GitOps Repositories}

Discussed by interview partner 2 and presented in section \ref{insights-related-ideas-approaches}
is the idea and problem, that when a human is given a GitOps repository,
it is often difficult to understand how the setup is exactly structured,
what the environments are,
what the versions are,
and what applications and version are deployed where.
The overview that GitOps shall give,
with the single place to look and know what is your system's state,
is not as good as it is expected,
interview partner 2 mentioned.

As possible future research on this idea,
this problem statement could be evaluated with a survey research against GitOps users.
In addition, especially if the survey's results speak for this statement,
a software tool could be developed,
which can recognize filesystem structures and plain text configuration files,
which represent the desired state.
This tool would have have knowledge of the different configuration management tools like Kustomize or Helm.
It would probably be beneficial to have a visual representation in the form of a dashboard.

Such a visual representation is already provided by ArgoCD for example,
however it will only show the configured data in the dashboard.
Any files, which are not yet added to the specific ArgoCD instance,
can only be viewed by having a look into the Git repository's filesystem manually.




\section*{Towards Standardized GitOps Promotions}

There is no standard way of doing promotions with the GitOps approach.
Sometimes a container image tag is changed in some place or places,
other times multiple files are copied over to another place (like the promotion process proposed with the prototype in this thesis),
and other times a promotion consists of multiple processes spanning over different domains.
This differs for each individual setup for distinctive organizations.

In order to get a better understanding of what the requirements are for different organizations,
and how they imagine a solution, and solved their individual use case,
it would be of use to survey a wide range and variety of organizations which adopted or want to adopt the GitOps approach.
For open-source tooling it is beneficial to strive for functionality that can be used by everyone,
instead of providing tailored tooling which may only work for a single setup at a particular organization.

















\chapter{Conclusion}
\label{conclusion}

TODO


%In einer Conclusio wird keine neue Information präsentiert, sondern das bereits gesagte nochmal zusammenfassend wiedergegeben. Du möchtest hier dem Leser die wichtigsten Key Points deiner Arbeit mitgeben, so dass er sich ewig dran erinnernt.
%
%In welchem Forschungsfeld sind wir nochmal und was ist das eigentlich Problem dort?
%
%Welcher Lösungsvorschlag wurde präsentiert, um dieses Problem zu lösen und welche Methodik wird eingesetzt, um das zu schaffen?
%
%Ausblick...wenn wir das dann geschafft haben, was können wir dann mehr, was wir heute nicht können?
%
%Future Work -> welche neuen Forschungsfelder eröffnen sich dann, wo können wir ab dann weitermachen?

\noindent
The currently available GitOps tools
do not provide an integrated solution to
the problem of promoting releases between environments.
This thesis aims at addressing the given problem.
This is achieved by
clearly defining the problem in distinct items,
from which research objectives are then inferred.
Following in designing and developing abstract models for
environments and promotion processes;
which are implemented in the produced artifact "release promotion operator".
The artifact is demonstrated in a prototype serving as a proof-of-concept.
Finally the research is evaluated by means of
comparing the artifact's functionality with the solution objectives.
\bigskip

\noindent
The developed "release promotion operator" prototype should serve as
a modular extension to Kubernetes and existing GitOps tooling within the CNCF.
It should provide a more streamlined and GitOps-native approach
to the process of release promotion.
For subsequent research projects the use of the prototype could
be observed in a case study or tested in a field experiment,
and adapted in another iteration of the applied methodology process model.
The main goal for future work is the extensive adaption of the prototype
to enable use in production by organizations or other projects.






























%
%\section{Instruction included in the original FHBgld word processor template}
%Die Zusammenfassung stellt die gesamte Arbeit – von der Einleitung bis zu den Ergebnissen - in Kurzform dar. Länge maximal 1,5 - 2 Seiten. 
%
%\section{Algorithms}
%
%If you want to show algorithms in your Thesis take a look at the \url{https://www.overleaf.com/learn/latex/algorithms} page. The \verb|algorithm2e| package is already included in the template. You can list algorithms in the same way as you can list Tables and Figures.
%
%\begin{algorithm}[H]
%	\KwData{this text}
%	\KwResult{how to write algorithm with \LaTeX2e }
%	initialization\;
%	\While{not at end of this document}{
%		read current\;
%		\eIf{understand}{
%			go to next section\;
%			current section becomes this one\;
%		}{
%			go back to the beginning of current section\;
%		}
%	}
%	\caption{How to write algorithms}
%\end{algorithm}
%%\cleardoublepage{}
%
%\section{Acronyms and Glossary.}
%
%if you want to use Acronyms or a Glossary check the page here: \url{https://www.overleaf.com/learn/latex/glossaries}
%
%The \Gls{latex} typesetting markup language is specially suitable 
%for documents that include \gls{maths}. are 
%rendered properly an easily once one gets used to the commands.
%
%Given a set of numbers, there are elementary methods to compute 
%its \glsxtrlong{gcd}, which is abbreviated \glsxtrshort{gcd}. This 
%process is similar to that used for the \glsxtrfull{lcm}.
%
%\section{Macros}
%
%You can also add useful packages or macros into the \verb|packages_macros.tex| file to add them to the project.
%The packages for algorithms, code or the glossary have already been added there.
%
%\subsection{FixMe}
%Another example, the \verb|FixMe| package, is added as well. It allows you or your supervisor to add Meta comments to the document. These comments only appear if you set the draft mode in the \verb|main.tex| file. If you remove or comment the activation of the draft mode you can see your final thesis without comments.