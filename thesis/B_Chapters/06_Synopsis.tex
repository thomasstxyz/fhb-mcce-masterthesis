\chapter{Synopsis}

TODO


%In einer Conclusio wird keine neue Information präsentiert, sondern das bereits gesagte nochmal zusammenfassend wiedergegeben. Du möchtest hier dem Leser die wichtigsten Key Points deiner Arbeit mitgeben, so dass er sich ewig dran erinnernt.
%
%In welchem Forschungsfeld sind wir nochmal und was ist das eigentlich Problem dort?
%
%Welcher Lösungsvorschlag wurde präsentiert, um dieses Problem zu lösen und welche Methodik wird eingesetzt, um das zu schaffen?
%
%Ausblick...wenn wir das dann geschafft haben, was können wir dann mehr, was wir heute nicht können?
%
%Future Work -> welche neuen Forschungsfelder eröffnen sich dann, wo können wir ab dann weitermachen?

\noindent
The currently available GitOps tools
do not provide an integrated solution to
the problem of promoting releases between environments.
This thesis aims at addressing the given problem.
This is achieved by
clearly defining the problem in distinct items,
from which research objectives are then inferred.
Following in designing and developing abstract models for
environments and promotion processes;
which are implemented in the produced artifact "release promotion operator".
The artifact is demonstrated in a prototype serving as a proof-of-concept.
Finally the research is evaluated by means of
comparing the artifact's functionality with the solution objectives.
\bigskip

\noindent
The developed "release promotion operator" prototype should serve as
a modular extension to Kubernetes and existing GitOps tooling within the CNCF.
It should provide a more streamlined and GitOps-native approach
to the process of release promotion.
For subsequent research projects the use of the prototype could
be observed in a case study or tested in a field experiment,
and adapted in another iteration of the applied methodology process model.
The main goal for future work is the extensive adaption of the prototype
to enable use in production by organizations or other projects.














\section{Instruction included in the original FHBgld word processor template}
Die Zusammenfassung stellt die gesamte Arbeit – von der Einleitung bis zu den Ergebnissen - in Kurzform dar. Länge maximal 1,5 - 2 Seiten. 

\section{Algorithms}

If you want to show algorithms in your Thesis take a look at the \url{https://www.overleaf.com/learn/latex/algorithms} page. The \verb|algorithm2e| package is already included in the template. You can list algorithms in the same way as you can list Tables and Figures.

\begin{algorithm}[H]
	\KwData{this text}
	\KwResult{how to write algorithm with \LaTeX2e }
	initialization\;
	\While{not at end of this document}{
		read current\;
		\eIf{understand}{
			go to next section\;
			current section becomes this one\;
		}{
			go back to the beginning of current section\;
		}
	}
	\caption{How to write algorithms}
\end{algorithm}
%\cleardoublepage{}

\section{Acronyms and Glossary.}

if you want to use Acronyms or a Glossary check the page here: \url{https://www.overleaf.com/learn/latex/glossaries}

The \Gls{latex} typesetting markup language is specially suitable 
for documents that include \gls{maths}. are 
rendered properly an easily once one gets used to the commands.

Given a set of numbers, there are elementary methods to compute 
its \glsxtrlong{gcd}, which is abbreviated \glsxtrshort{gcd}. This 
process is similar to that used for the \glsxtrfull{lcm}.

\section{Macros}

You can also add useful packages or macros into the \verb|packages_macros.tex| file to add them to the project.
The packages for algorithms, code or the glossary have already been added there.

\subsection{FixMe}
Another example, the \verb|FixMe| package, is added as well. It allows you or your supervisor to add Meta comments to the document. These comments only appear if you set the draft mode in the \verb|main.tex| file. If you remove or comment the activation of the draft mode you can see your final thesis without comments.