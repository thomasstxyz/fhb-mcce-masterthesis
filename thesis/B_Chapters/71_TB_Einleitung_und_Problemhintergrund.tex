\chapter{Einleitung und Problemhintergrund}

% Forschungsfeld (Universum), 5-7 Sätze
% TODO: wie kommt man zu GitOps? 1,2 Sätze
% universum -> erde -> kontinent -> land -> region -> strand -> sandkorn

% Was ist denn das überhaupt
% Was ist GitOps

GitOps ist eine Reihe von Prinzipien für den Betrieb und die Verwaltung von Softwaresystemen. Diese Prinzipien sind aus dem modernen Softwarebetrieb abgeleitet, haben aber auch ihre Wurzeln in bereits bestehenden und weithin angenommenen Best Practices. Die primären vier Prinzipien, welche als Grundsätze für den gewünschten Zustand eines
von GitOps verwalteten System gelten, sind die folgenden:

\begin{itemize}
	\item \textbf{Deklarativ} \\
		Bei einem von GitOps verwalteten System muss der gewünschte Zustand deklarativ ausgedrückt werden.
	\item \textbf{Versioniert und unveränderlich} \\
		Der gewünschte Zustand wird auf eine Weise gespeichert, die Unveränderlichkeit und Versionierung erzwingt und eine vollständige Versionshistorie aufbewahrt.
	\item \textbf{Automatisch abgerufen} \\
		Software-Agenten ziehen automatisch die gewünschten Zustands-Deklarationen aus der Quelle.
	\item \textbf{Kontinuierlich abgeglichen} \\
		Software-Agenten beobachten kontinuierlich den aktuellen Systemzustand und versuchen, den gewünschten Zustand herzustellen.
\end{itemize}

\autocite{gitopsPrinciplesv100}
\bigskip

Diese Grundsätze sind in der Version 1.0.0 von OpenGitOps festgelegt.
OpenGitOps ist ein CNCF Sandbox-Projekt im Rahmen der GitOps Working Group.
Die GitOps Working Group ist eine Arbeitsgruppe unter
dem CNCF App Delivery TAG,
mit dem Ziel, eine klare herstellerneutrale,
prinzipiengeleitete Bedeutung von GitOps zu definieren.
Damit wird eine Grundlage für die Interoperabilität zwischen Tools, Konformität und Zertifizierung durch dauerhafte Programme, Dokumente und Code geschaffen
\autocite{opengitopsDocuments}.
\bigskip

Es ist anzumerken, dass sich der Begriff GitOps nicht ausschließlich auf
die in OpenGitOps definierten Grundsätze beschränkt,
sondern darüber hinaus geht.
% TODO: Quelle, wo GitOps noch definiert ist
% TODO: was ist im Groben der Unterschied zwischen OpenGitOps und generell GitOps?
\bigskip

GitOps senkt die Kosten für die Erstellung von Self-Service-IT-Systemen und ermöglicht Self-Service-Operationen, die bisher nicht zu rechtfertigen waren.
Es verbessert die Fähigkeit, das System sicher zu betreiben, indem es auch
unprivilegierten Nutzern erlaubt, große Änderungen vorzunehmen,
ohne diesen Nutzern Berechtigungen am Zielsystem zu erteilen.
Über sogenannte Pull Requests können gewöhnliche Nutzer einen Antrag
ihrer vorgeschlagenen Änderungen stellen.
Diese Pull Requests können im Anschluss von höher privilegierten Nutzern
überprüft und gegebenenfalls akzeptiert werden.
Die Sicherheit kann verbessert werden, 
indem automatisierte Tests hinzugefügt werden.
Diese Tests können beispielsweise
Linting (Validierung der Syntax),
das Erzwingen von Style-Guides,
das Durchführen von Unit Tests,
oder
das Durchsetzen von Sicherheitsrichtlinien
umfassen.
Sicherheitsprüfungen und Audits werden einfacher, 
da jede Änderung nachverfolgt wird
\autocite{limoncelli_gitopsPathToMoreSelfService}.
\bigskip
% TODO: Idee 1
% es gibt derzeit diese Probleme mit GitOps

GitOps als Praxis für das Releasen von Software hat mehrere Vorteile, aber wie alle anderen Lösungen davor, hat GitOps auch mehrere Unzulänglichkeiten.
Derzeit gibt es mit GitOps einige ungelöste Probleme.
Einserseits sind das Probleme, welche meist schon 
vor der Implementierung von GitOps in einer Organisation
ein Problem waren,
andererseits bringt die GitOps-Strategie einige Limitierungen mit sich,
welche zu neuen Problemen führen.
Die Organisation Codefresh,
welche aufgrund ihrer Entwicklungen
von GitOps-Tools um ArgoCD und verwandte Argo*-Produkte,
als treibende Kraft für GitOps gilt,
spricht von folgenden Problemen mit dem derzeitigen praktizierten GitOps:

\begin{itemize}
	\item GitOps deckt nur eine Teilmenge des Software-Lebenszyklus ab
	\item Die Aufteilung von CI und CD mit GitOps ist nicht ganz einfach
	\item GitOps befasst sich nicht mit der Promotion von Releases zwischen Umgebungen
	\item Es gibt keine Standardpraxis für die Modellierung von Konfigurationen mit mehreren Umgebungen
	\item GitOps versagt bei automatischer Skalierung und dynamischen Ressourcen
	\item Es gibt keine Standardpraxis für GitOps-Rollbacks
	\item Die Observability für GitOps (und Git) ist unausgereift
	\item Auditing ist problematisch, obwohl alle Informationen in Git vorhanden sind
	\item Der Betrieb von GitOps in großem Maßstab ist schwierig
	\item GitOps und Helm arbeiten nicht immer gut zusammen
	\item Continuous Deployment und GitOps passen nicht zusammen
	\item Es gibt keine Standardpraxis für die Verwaltung von Secrets
\end{itemize}

\autocite{codefreshGitopsPains10}
\bigskip

% TODO: außerdem werden von Author/Org XY folgende Probleme hervorgehoben..


% Ich nehme mir dieses eine Problem heraus

In dieser Arbeit soll das Problem der
Promotion von Releases zwischen Umgebungen in GitOps
näher betrachtet werden.
Dabei handelt es sich insbesondere um
die Promotion bzw. Beförderung von neuen Releases
von einer Deployment-Umgebung in eine andere bzw. in die nächste.
\bigskip

Vor allem große Organisationen verfügen für gewöhnlich über mehrere
Nicht-Produktions- und Produktions-Umgebungen
wie z. B. QA, Staging, Production.
In der Regel werden neue Releases automatisch in einer Umgebung
wie beispielsweise QA ausgerollt.
Nun ist die Aufgabe,
neue Änderungen, welche durch ein neues Release herbeigeführt werden,
in nachfolgende bzw. andere Umgebungen zu befördern.
Bisher fehlt dafür jegliche Standardpraxis.
\bigskip

Insbesondere sollen in dieser Arbeit
bestehende Strategien für die Lösung des Problems
mit existierenden Tools
erforscht werden.
Des Weiteren soll ein Prototyp einer neu entwickelten
Strategie vorgeschlagen werden.
Im Anschluss soll der entwickelte Prototyp
im Rahmen eines Laborexperiments
mit den bereits existierenden Strategien
verglichen werden.
\bigskip

Der entwickelte Prototyp soll als Ziel haben,
ein GitOps-Tool bereitzustellen,
das die Promotion von Releases zwischen Umgebungen
nach den GitOps-Prinzipien ermöglicht
in bestehende GitOps-Workflows nach 
den Best Practices
\bigskip

Die erforschten Ergebnisse,
den erarbeiteten Lösungsansatz,
sowie der vorgeschlagene Prototyp,
können zum Ende
an die CNCF gespendet werden.

% und dies ist meine Idee zu diesem Problem

% TODO: Idee 1



% TODO: Idee 2

% TODO: Idee x





% TODO: Zusammenfassung

% TODO: Problemstellung, 5-7 Sätze

% TODO: Forschungsvorhaben, 5-7 Sätze

% TODO: Ausblick, 3-5 Sätze





