\chapter{Introduction}

% Forschungsfeld (Universum), 5-7 Sätze
% wie kommt man zu GitOps? 1,2 Sätze
% universum -> erde -> kontinent -> land -> region -> strand -> sandkorn

Increasingly more organizations are adopting 
a DevOps culture to develop new applications and services at high velocity. 
After all, a culture that encourages shared responsibility, transparency and rapid feedback, 
helps narrow the gaps between teams and thus accelerate processes.
GitOps was born out of the need for rapid innovation.
\bigskip

% Was ist denn das überhaupt
% Was ist GitOps

\noindent
GitOps is a set of principles for operating and managing software systems.
These principles are derived from modern software operations, but also have their roots 
in existing and widely adopted best practices. The primary four principles,
which serve as principles for the desired state of a
system managed by GitOps are the following:

\begin{itemize}
	\item \textbf{Declarative} \\
		A system managed by GitOps must have its desired state expressed declaratively.
	\item \textbf{Versioned and Immutable} \\
		Desired state is stored in a way that enforces immutability, versioning and retains a complete version history.
	\item \textbf{Pulled Automatically} \\
		Software agents automatically pull the desired state declarations from the source.
	\item \textbf{Continuously Reconciled} \\
		Software agents continuously observe actual system state and attempt to apply the desired state.
\end{itemize}

\noindent
\autocite{gitopsPrinciplesv100}
\bigskip

\noindent
These principles are defined in OpenGitOps version 1.0.0.
OpenGitOps is a Cloud Native Computing Foundation (CNCF) sandbox project under the GitOps Working Group.
The GitOps Working Group is a working group under
the CNCF App Delivery TAG,
with the goal of establishing a clear vendor-neutral,
principle-driven meaning of GitOps.
This will provide a foundation for interoperability between tools, conformance and certification through enduring programs, documents and code
\autocite{opengitopsDocuments}.
%\bigskip
%
%\noindent
It should be noted that the term GitOps is not exclusively limited to
the principles defined in OpenGitOps,
but goes beyond that.
% TODO: Quelle, wo GitOps noch definiert ist
% TODO: was ist im Groben der Unterschied zwischen OpenGitOps und generell GitOps?
\bigskip

%\noindent
%GitOps can be viewed as an evolution of Infrastructure as Code (IaC) that uses Git as a version control system for infrastructure configurations.
%GitOps lowers the cost of building self-service IT systems and enables self-service operations that were previously unjustifiable
%\autocite{limoncelli_gitopsPathToMoreSelfService}.
%It improves the ability to operate systems securely by
%allowing unprivileged users to make major changes,
%without granting these users direct access to the target system.
%Through so-called pull requests, ordinary users can submit a request
%of their proposed changes.
%These pull requests can then be reviewed and accepted by more privileged users.
%If necessary, adjustments can be suggested or incorporated by the reviewers themselves.
%\bigskip
%
%\noindent
%Security can be improved by adding automated tests.
%These tests can include
%Linting (validation of syntax),
%enforcing style guides,
%running end-to-end and unit tests,
%or
%enforcing security policies.
%Security reviews and audits become easier 
%because every change is tracked.
%Every change or commit is permanently recorded in the Git history
%\autocite{limoncelli_gitopsPathToMoreSelfService}.
%\bigskip

% es gibt derzeit diese Probleme mit GitOps

\noindent
GitOps as a practice for releasing software has several advantages,
but like other solutions, GitOps also has several shortcomings.
Currently, there are several unresolved problems with GitOps.
On the one hand, these are problems that often 
were a problem in an organization already before GitOps was implemented,
on the other hand, a GitOps strategy brings with it some limitations,
which lead to new problems.
Depending on the level of adoption of GitOps,
and how strictly the principles are followed,
some problems may or may not be an issue for an organization.
\bigskip
% TODO: über gitops at scale auf das promotion problem hinweisen ...

\noindent
When considering GitOps at scale,
adopters of GitOps and its current tools and practices
soon realize the problem of
modeling multiple environments
and promoting releases between them.
Current tools do not come with an out-of-the-box solution
for describing multiple environments.
It is to decide by the user
how to model their multiple environments in a GitOps fashion.
There have emerged a couple of models for describing environments
like branch-per-environment,
folder-per-environment, repo-per-environment,
which all seem to have their individual upsides and downsides.
For every approach, the process of promotion is different.
Clear guidelines and best practices,
as well as tools which implement them,
are missing in the GitOps ecosystem.








%The organization Codefresh
%%Codefresh is the organization behind the Argo Project (argoproj.io, 2023), which is developing a suite of open source tools for GitOps
%%\autocite{argoProjWebsite}.
%as a driving force for GitOps,
%- due to its developments
%of various GitOps tools -
%speaks of the following problems with the currently practiced GitOps:
%
%\begin{itemize}
%	\item GitOps covers only a subset of the software lifecycle
%	\item Splitting CI and CD with GitOps is not straightforward
%	\item GitOps doesn’t address promotion of releases between environments
%	\item There is no standard practice for modeling multi-environment configurations
%	\item GitOps breaks down with auto-scaling and dynamic resources
%	\item There is no standard practice for GitOps rollbacks
%	\item Observability for GitOps (and Git) is immature
%	\item Auditing is problematic despite having all information in Git
%	\item Running GitOps at scale is difficult
%	\item GitOps and Helm do not always work well together
%	\item Continuous Deployment and GitOps do not mix together
%	\item There is no standard practice for managing secrets
%\end{itemize}
%
%\noindent
%\autocite{codefreshGitopsPains10}
%\bigskip
%
%% immer fließtext nach aufzählungen, tabellen, etc.
%\noindent
%The listed problems may not apply to every organization using GitOps.
%Depending on the level of adoption of GitOps,
%and how strictly the principles are followed,
%the points listed may cause more or less troubles.




