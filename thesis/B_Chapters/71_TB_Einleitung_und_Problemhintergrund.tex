\chapter{Introduction}

\noindent
TODO intro umschreiben in 4 Absätze

% 1. Absatz: Beschreibe uns das Forschungsfeld in dem du dich befindest (DevOps / GitOps)

% 2. Absatz: Beschreib uns das Problem bzw. die Challenge in diesem Forschungsfeld (Was ist die Problemstellung, die es zu bearbeiten gilt)

% 3. Absatz: Was schlägst du vor, wie man die Problemstellung bearbeiten könnte? Beschreibe uns deinen Lösugnsvorschlag dieser Arbeit, wie man das Problem bearbeiten könnte

% 4. Absatz: gib uns einen Ausblick (Paint the Big Picture) wie deine Lösung mit dem großen Ganzen zusammenhängt


% am Ende der alten Intro...
% hier fehlt mir eine klare Beschreibung der Problemstellung...du bist sehr generisch und sagst, dass es vor und nachteile hat, benennst diese aber nicht
% Was auch fehlt ist dein Lösungsvorschlag bzw. deine Idee, wie man die Problemstellung bearbeiten kann (was du in dieser Arbeit vorhast)





\bigskip

% Forschungsfeld (Universum), 5-7 Sätze
% wie kommt man zu GitOps? 1,2 Sätze
% universum -> erde -> kontinent -> land -> region -> strand -> sandkorn

Increasingly more organizations are adopting 
a DevOps culture to develop new applications and services at high velocity. 
After all, a culture that encourages shared responsibility, transparency and rapid feedback, 
helps to narrow the gaps between teams and thus accelerate the development process.
In order to
reduce friction between engineering teams who are involved in the software development lifecycle (SDLC),
a new practice called GitOps has emerged.
It allows developers who are already familiar with the revision control system Git,
to easily deploy their applications to target environments in a self-service model.
System administrators and operators can also manage IT infrastructure
purely by interfacing with declarative state definitions stored in Git.
% GitOps
GitOps is a set of principles for operating and managing software systems.
These principles are derived from modern software operations, but also have their roots 
in existing and widely adopted best practices. The primary four principles,
which serve as principles for the desired state of a
system managed by GitOps are the following \autocite{gitopsPrinciplesv100}:

\begin{itemize}
	\item \textbf{Declarative} \\
		A system managed by GitOps must have its desired state expressed declaratively.
	\item \textbf{Versioned and Immutable} \\
		Desired state is stored in a way that enforces immutability, versioning and retains a complete version history.
	\item \textbf{Pulled Automatically} \\
		Software agents automatically pull the desired state declarations from the source.
	\item \textbf{Continuously Reconciled} \\
		Software agents continuously observe actual system state and attempt to apply the desired state.
\end{itemize}

% TODO: ensure list does not lap over multiple pages in PDF export

\noindent
These principles are defined in OpenGitOps version 1.0.0,
which is a Cloud Native Computing Foundation (CNCF) sandbox project
in the App Delivery TAG 
under the GitOps Working Group.
The overall goal of OpenGitOps is to establish a clear vendor-neutral,
principle-driven meaning of GitOps,
which shall provide a foundation for interoperability between tools, conformance and certification through enduring programs, documents and code
\autocite{opengitopsDocuments}.
This thesis aims at adhering to the definitions in OpenGitOps
for the term GitOps, its principles and the glossary around it.
However, the research is not limited to the definitions in OpenGitOps,
and might even suggest changes.


% However, since OpenGitOps is still in the sandbox phase,
% the definitions are likely to change.
% Furthermore, this research is not limited to the OpenGitOps initiative,
% and might introduce new concepts or adaptions and suggest them.

% The term GitOps is not exclusively limited to
% the principles defined in OpenGitOps,
% but goes beyond that.
% TODO: Quelle, wo GitOps noch definiert ist
% TODO: was ist im Groben der Unterschied zwischen OpenGitOps und generell GitOps?
% Since the concept of GitOps itself is still quite new,
% the definitions in OpenGitOps are still open to change
% \bigskip

%\noindent
%GitOps can be viewed as an evolution of Infrastructure as Code (IaC) that uses Git as a version control system for infrastructure configurations.
%GitOps lowers the cost of building self-service IT systems and enables self-service operations that were previously unjustifiable
%\autocite{limoncelli_gitopsPathToMoreSelfService}.
%It improves the ability to operate systems securely by
%allowing unprivileged users to make major changes,
%without granting these users direct access to the target system.
%Through so-called pull requests, ordinary users can submit a request
%of their proposed changes.
%These pull requests can then be reviewed and accepted by more privileged users.
%If necessary, adjustments can be suggested or incorporated by the reviewers themselves.
%\bigskip
%
%\noindent
%Security can be improved by adding automated tests.
%These tests can include
%Linting (validation of syntax),
%enforcing style guides,
%running end-to-end and unit tests,
%or
%enforcing security policies.
%Security reviews and audits become easier 
%because every change is tracked.
%Every change or commit is permanently recorded in the Git history
%\autocite{limoncelli_gitopsPathToMoreSelfService}.
%\bigskip

% es gibt derzeit diese Probleme mit GitOps

\noindent
GitOps as a practice for releasing software has many advantages,
but like other solutions, GitOps also has some shortcomings.
One of the unresolved problems is
the process of promoting releases between multiple deployment environments (illustrated in fig. \ref{fig:releasePromotionProcess}).
%% why is promotion important?
In recent years with DevOps and Continuous Delivery practices,
the release promotion process through different environments is becoming ever more important.
Containerization technologies allow for easy deployments to
uniquely different environments.
However it is important to model different environments in
a homogeneous way,
in order to simplify the full automation of the software delivery process.

\begin{figure}[h]
	\centering
	\includegraphics[width=.55\linewidth]{figures/release-promotion.drawio.png}
	\caption{release promotion process.
%		(\citeauthor{ref}, \citeyear{ref}).
	}
	\label{fig:releasePromotionProcess}	
\end{figure}

\noindent
Previously, this process was usually handled by the CI/CD system,
which would execute a pipeline, in which every environment
is delivered to consecutively
in an imperative way.
With GitOps, the state of the different environments
is completely defined
in a declarative way.
The system state is then continuously reconciled
by the GitOps controller.
What turns out to be tricky,
is the process of promotion to other deployment environments.
Current GitOps tools do not provide an integrated solution for this process,
nor do they provide any sort of abstraction for defining environments.
Users currently need to rely on separation of duties
on a file and folder level within the Git repository,
for modeling different environments.
Promotions are often achieved via hard-coded file copy operations,
which is done by the CI/CD system.
In addition, for each configuration or templating tool which is used
(e. g. kustomize, helm, jsonnet, etc.),
the modeling of different environments, as well as the
process of promotion, is unique.
This makes the process rather difficult.
Clear guidelines and best practices,
as well as tools which implement them,
are missing in the GitOps ecosystem.
\bigskip

\noindent
In order to help with full automation of the 
release and promotion process of new software releases,
this gap needs to be filled.
Not only larger organizations need a high level of automation.
Smaller organizations also
greatly benefit from automation, in order
to deliver high quality software.
\bigskip





%\noindent
%When considering GitOps at scale,
%adopters of GitOps and its current tools and practices
%soon realize the problem of
%modeling multiple environments
%and promoting releases between them.
%Current tools do not come with an out-of-the-box solution
%for describing multiple environments.
%It is to decide by the user
%how to model their multiple environments in a GitOps fashion.
%There have emerged a couple of models for describing environments
%like branch-per-environment,
%folder-per-environment, repo-per-environment,
%which all seem to have their individual upsides and downsides.
%For every approach, the process of promotion is different.





%The organization Codefresh
%%Codefresh is the organization behind the Argo Project (argoproj.io, 2023), which is developing a suite of open source tools for GitOps
%%\autocite{argoProjWebsite}.
%as a driving force for GitOps,
%- due to its developments
%of various GitOps tools -
%speaks of the following problems with the currently practiced GitOps:
%
%\begin{itemize}
%	\item GitOps covers only a subset of the software lifecycle
%	\item Splitting CI and CD with GitOps is not straightforward
%	\item GitOps doesn’t address promotion of releases between environments
%	\item There is no standard practice for modeling multi-environment configurations
%	\item GitOps breaks down with auto-scaling and dynamic resources
%	\item There is no standard practice for GitOps rollbacks
%	\item Observability for GitOps (and Git) is immature
%	\item Auditing is problematic despite having all information in Git
%	\item Running GitOps at scale is difficult
%	\item GitOps and Helm do not always work well together
%	\item Continuous Deployment and GitOps do not mix together
%	\item There is no standard practice for managing secrets
%\end{itemize}
%
%\noindent
%\autocite{codefreshGitopsPains10}
%\bigskip
%
%% immer fließtext nach aufzählungen, tabellen, etc.
%\noindent
%The listed problems may not apply to every organization using GitOps.
%Depending on the level of adoption of GitOps,
%and how strictly the principles are followed,
%the points listed may cause more or less troubles.




