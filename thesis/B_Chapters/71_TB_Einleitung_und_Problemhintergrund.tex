\chapter{Einleitung und Problemhintergrund}

% Forschungsfeld (Universum), 5-7 Sätze
% wie kommt man zu GitOps? 1,2 Sätze
% universum -> erde -> kontinent -> land -> region -> strand -> sandkorn

Immer mehr Organisationen setzen auf 
eine DevOps-Kultur, um neue Applikationen und Dienste mit hoher Geschwindigkeit zu entwickeln. 
Denn eine Kultur, die gemeinsame Verantwortung, Transparenz und schnelles Feedback fördert, 
hilft dabei, die Lücken zwischen Teams zu verkleinern und damit auch Prozesse zu beschleunigen.
Aus dem Bedürfnis nach schneller Innovation entstand GitOps.
\bigskip

% Was ist denn das überhaupt
% Was ist GitOps

\noindent
GitOps ist eine Reihe von Prinzipien für den Betrieb und die Verwaltung von Softwaresystemen.
Diese Prinzipien sind aus dem modernen Softwarebetrieb abgeleitet, haben aber auch ihre Wurzeln 
in bereits bestehenden und weithin angenommenen Best Practices. Die primären vier Prinzipien,
welche als Grundsätze für den gewünschten Zustand eines
von GitOps verwalteten System gelten, sind die folgenden:

\begin{itemize}
	\item \textbf{Deklarativ} \\
		Bei einem von GitOps verwalteten System muss der gewünschte Zustand deklarativ ausgedrückt werden.
	\item \textbf{Versioniert und unveränderlich} \\
		Der gewünschte Zustand wird auf eine Weise gespeichert, die Unveränderlichkeit und Versionierung erzwingt und eine vollständige Versionshistorie aufbewahrt.
	\item \textbf{Automatisch abgerufen} \\
		Software-Agenten ziehen automatisch die gewünschten Zustands-Deklarationen aus der Quelle.
	\item \textbf{Kontinuierlich abgeglichen} \\
		Software-Agenten beobachten kontinuierlich den aktuellen Systemzustand und versuchen, den gewünschten Zustand herzustellen.
\end{itemize}

\noindent
\autocite{gitopsPrinciplesv100}
\bigskip

\noindent
Diese Grundsätze sind in der Version 1.0.0 von OpenGitOps festgelegt.
OpenGitOps ist ein Cloud Native Computing Foundation (CNCF) Sandbox-Projekt im Rahmen der GitOps Working Group.
Die GitOps Working Group ist eine Arbeitsgruppe unter
dem CNCF App Delivery TAG,
mit dem Ziel, eine klare herstellerneutrale,
prinzipiengeleitete Bedeutung von GitOps zu definieren.
Damit wird eine Grundlage für die Interoperabilität zwischen Tools, Konformität und Zertifizierung durch dauerhafte Programme, Dokumente und Code geschaffen
\autocite{opengitopsDocuments}.
\bigskip

\noindent
Es ist anzumerken, dass sich der Begriff GitOps nicht ausschließlich auf
die in OpenGitOps definierten Grundsätze beschränkt,
sondern darüber hinaus geht.
% TODO: Quelle, wo GitOps noch definiert ist
% TODO: was ist im Groben der Unterschied zwischen OpenGitOps und generell GitOps?
\bigskip

\noindent
GitOps kann als eine Weiterentwicklung von Infrastructure as Code (IaC) betrachtet werden, die Git als Versionskontrollsystem für Infrastrukturkonfigurationen verwendet.
GitOps senkt die Kosten für die Erstellung von Self-Service-IT-Systemen und ermöglicht Self-Service-Operationen, die bisher nicht zu rechtfertigen waren.
Es verbessert die Fähigkeit, Systeme sicher zu betreiben, indem es
unprivilegierten Nutzern erlaubt, große Änderungen vorzunehmen,
ohne diesen Nutzern direkten Zugriff am Zielsystem zu erteilen.
Über sogenannte Pull Requests können gewöhnliche Nutzer einen Antrag
ihrer vorgeschlagenen Änderungen stellen.
Diese Pull Requests können im Anschluss von höher privilegierten Nutzern
überprüft und akzeptiert werden.
Gegebenenfalls können Anpassungen vorgeschlagen, oder selbst eingearbeitet werden.
\bigskip

\noindent
Die Sicherheit kann verbessert werden, indem automatisierte Tests hinzugefügt werden.
Diese Tests können beispielsweise
Linting (Validierung der Syntax),
das Erzwingen von Style-Guides,
das Durchführen von Unit Tests,
oder
das Durchsetzen von Sicherheitsrichtlinien
umfassen.
Sicherheitsprüfungen und Audits werden einfacher, 
da jede Änderung nachverfolgt wird.
Jede Änderung bzw. Commit wird permanent in der Git-Historie festgehalten
\autocite{limoncelli_gitopsPathToMoreSelfService}.
\bigskip

% es gibt derzeit diese Probleme mit GitOps

\noindent
GitOps als Praxis für das Releasen von Software hat mehrere Vorteile,
aber wie auch andere Lösungen, hat GitOps auch mehrere Unzulänglichkeiten.
Derzeit gibt es mit GitOps einige ungelöste Probleme.
Einserseits sind das Probleme, welche meist schon 
vor der Implementierung von GitOps in einer Organisation ein Problem waren,
andererseits bringt eine GitOps-Strategie einige Limitierungen mit sich,
welche zu neuen Problemen führen.
Die Organisation Codefresh,
welche aufgrund ihrer Entwicklungen
von diversen GitOps-Tools
als treibende Kraft für GitOps gilt,
spricht von folgenden Problemen mit dem zurzeit praktizierten GitOps:

\begin{itemize}
	\item GitOps deckt nur eine Teilmenge des Software-Lebenszyklus ab
	\item Die Aufteilung von CI und CD mit GitOps ist nicht ganz einfach
	\item GitOps befasst sich nicht mit der Promotion von Releases zwischen Umgebungen
	\item Es gibt keine Standardpraxis für die Modellierung von Konfigurationen mit mehreren Umgebungen
	\item GitOps versagt bei automatischer Skalierung und dynamischen Ressourcen
	\item Es gibt keine Standardpraxis für GitOps-Rollbacks
	\item Die Observability für GitOps (und Git) ist unausgereift
	\item Auditing ist problematisch, obwohl alle Informationen in Git vorhanden sind
	\item Der Betrieb von GitOps in großem Maßstab ist schwierig
	\item GitOps und Helm arbeiten nicht immer gut zusammen
	\item Continuous Deployment und GitOps passen nicht zusammen
	\item Es gibt keine Standardpraxis für die Verwaltung von Secrets
\end{itemize}

\noindent
\autocite{codefreshGitopsPains10}
\bigskip

% TODO: außerdem werden von Author/Org XY folgende Probleme hervorgehoben..



\noindent
Die erforschten Ergebnisse,
den erarbeiteten Lösungsansatz,
sowie der vorgeschlagene Prototyp,
können zum Ende
an die CNCF gespendet werden.
