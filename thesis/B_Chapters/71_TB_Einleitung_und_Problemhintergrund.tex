\chapter{Einleitung und Problemhintergrund}

% Forschungsfeld (Universum), 5-7 Sätze
% TODO: wie kommt man zu GitOps? 1,2 Sätze
% universum -> erde -> kontinent -> land -> region -> strand -> sandkorn

% Was ist denn das überhaupt
% Was ist GitOps

GitOps ist eine Reihe von Prinzipien für den Betrieb und die Verwaltung von Softwaresystemen. Diese Prinzipien sind aus dem modernen Softwarebetrieb abgeleitet, haben aber auch ihre Wurzeln in bereits bestehenden und weithin angenommenen Best Practices. Die primären vier Prinzipien, welche als Grundsätze für den gewünschten Zustand eines
von GitOps verwalteten System gelten, sind die folgenden:

\begin{itemize}
	\item \textbf{Deklarativ} \\
		Bei einem von GitOps verwalteten System muss der gewünschte Zustand deklarativ ausgedrückt werden.
	\item \textbf{Versioniert und unveränderlich} \\
		Der gewünschte Zustand wird auf eine Weise gespeichert, die Unveränderlichkeit und Versionierung erzwingt und eine vollständige Versionshistorie aufbewahrt.
	\item \textbf{Automatisch abgerufen} \\
		Software-Agenten ziehen automatisch die gewünschten Zustands-Deklarationen aus der Quelle.
	\item \textbf{Kontinuierlich abgeglichen} \\
		Software-Agenten beobachten kontinuierlich den aktuellen Systemzustand und versuchen, den gewünschten Zustand herzustellen.
\end{itemize}

\autocite{gitopsPrinciplesv100}
\bigskip

Diese Grundsätze sind in der Version 1.0.0 von OpenGitOps festgelegt.
OpenGitOps ist ein CNCF Sandbox-Projekt im Rahmen der GitOps Working Group.
Die GitOps Working Group ist eine Arbeitsgruppe unter
dem CNCF App Delivery TAG,
mit dem Ziel, eine klare herstellerneutrale,
prinzipiengeleitete Bedeutung von GitOps zu definieren.
Damit wird eine Grundlage für die Interoperabilität zwischen Tools, Konformität und Zertifizierung durch dauerhafte Programme, Dokumente und Code geschaffen
\autocite{opengitopsDocuments}.
\bigskip

Es ist anzumerken, dass sich der Begriff GitOps nicht ausschließlich auf
die in OpenGitOps definierten Grundsätze beschränkt,
sondern darüber hinaus geht.
% TODO: Quelle, wo GitOps noch definiert ist
% TODO: was ist im Groben der Unterschied zwischen OpenGitOps und generell GitOps?
\bigskip

GitOps senkt die Kosten für die Erstellung von Self-Service-IT-Systemen und ermöglicht Self-Service-Operationen, die bisher nicht zu rechtfertigen waren.
Es verbessert die Fähigkeit, das System sicher zu betreiben, indem es auch
unprivilegierten Nutzern erlaubt, große Änderungen vorzunehmen,
ohne diesen Nutzern Berechtigungen am Zielsystem zu erteilen.
Über sogenannte Pull Requests können gewöhnliche Nutzer einen Antrag
ihrer vorgeschlagenen Änderungen stellen.
Diese Pull Requests können im Anschluss von höher privilegierten Nutzern
überprüft und gegebenenfalls akzeptiert werden.
Die Sicherheit kann verbessert werden, 
indem automatisierte Tests hinzugefügt werden.
Diese Tests können beispielsweise
Linting (Validierung der Syntax),
das Erzwingen von Style-Guides,
das Durchführen von Unit Tests,
oder
das Durchsetzen von Sicherheitsrichtlinien
umfassen.
Sicherheitsprüfungen und Audits werden einfacher, 
da jede Änderung nachverfolgt wird
\autocite{limoncelli_gitopsPathToMoreSelfService}.



% TODO: Idee 1

% TODO: Idee 2

% TODO: Idee 3

% TODO: Idee xy

% TODO: Zusammenfassung

% TODO: Problemstellung, 5-7 Sätze

% TODO: Forschungsvorhaben, 5-7 Sätze

% TODO: Ausblick, 3-5 Sätze





