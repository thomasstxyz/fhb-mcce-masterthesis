\chapter{Einleitung und Problemhintergrund}

Im Bereich der Containertechnologien zeigt das Cloud-Computing ein rasantes Wachstum aufgrund der
vielen Vorteile, die mit container-basierter Softwarearchitektur einhergehen.
Immer mehr Softwarehersteller setzen auf cloud-native Architektur,
meist basierend auf der open-source Container-Plattform Kubernetes.
Kubernetes ist unter der Governance der Cloud Native Computing Foundation (CNCF) seit 10. März 2016
und verfügt über den Reifegrad \emph{graduated}.
\bigskip

Mit Kubernetes lassen sich als Container abgepackte Microservices im großen Maßstab auf
mehreren tausend Nodes orchestrieren.
Das Monitoring zielt derzeit hauptsächlich auf Metriken der CPU und Arbeitsspeicher-Auslastung ab.
Dabei können für jeden Kubernetes Workload diese Metriken gemessen und somit fein-granular überwacht werden,
sowie im Anschluss weitere Automatisierung, basierend auf diversen Service Level Objectives (SLO) stattfinden,
oder horizontales und vertikales Auto-Scaling konfiguriert werden, neben anderen Automatisierungen.
\bigskip

Derzeit kann der Stromverbrauch nur je Node gemessen werden.
Das Problem ist, dass in Kubernetes ein Microservice bzw. Workload nicht mehr wie früher
auf einer, oder einer Handvoll physischen oder virtuellen Maschinen betrieben wird,
sondern sich über potenziell tausende Nodes verteilt.
Kubernetes Workloads sind in sogenannte Pods logisch in kleine Einheiten aufgeteilt.
Im Optimalfall enkapsuliert ein Pod einen oder mehrere verwandte Container, in welcher jeweils ein Prozess läuft.
Somit ist es äußerst schwierig, den Stromverbrauch einer Applikation festzustellen.
\bigskip

In dieser Arbeit soll die Messung des Stromverbrauchs auf Ebene von Kubernetes Workloads erforscht werden.
Dies würde es beispielsweise erlauben, einen stromverbrauchs-orientierten Kubernetes-nativen
Scheduler zu implementieren, welcher z. B. Nodes für Workloads adaptiv bestimmt,
je nachdem auf welchen Node-Typen ein bestimmter Workload die beste Energieeffizienz aufweist.
Node-Typen können sich beispielsweise unterscheiden durch:
CPU-Architekturen, Massenspeicher-Typen, Netzwerkanbindungen, Netzwerklatenz, Geo-Lokation, etc.
\bigskip

%Zusammenfassung



%Problemstellung, 5-7 Sätze

%Forschungsvorhaben, 5-7 Sätze

%Ausblick, 3-5 Sätze
