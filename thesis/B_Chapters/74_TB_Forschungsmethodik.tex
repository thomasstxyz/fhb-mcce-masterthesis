\chapter{Forschungsmethodik}

Zuerst wird bestehende Literatur zum Thema, 
sowie der konkreten Forschungsfragen herangezogen.
Des Weiteren werden
diverse Blogbeiträge von wegweisenden Organisationen im GitOps-Umfeld analysiert
und die darin empfohlenen Best Practices sowie vorgestellten Modelle näher betrachtet.
\bigskip

\noindent
Als nächstes werden existierende Softwarelösungen,
die unter Open-Source-Lizenzen zur Verfügung stehen,
für den zweckmäßigen Einsatz für die vorliegende Forschung evaluiert.
Es gilt herauszufinden, inwieweit die definierten Forschungsfragen
mit bereits existierenden Lösungen
addressiert werden können.
Schließlich wird eine Baseline als State-of-the-Art definiert.
\bigskip

\noindent
Weiters sollen abstrakte Modelle für Deployment-Umgebungen
sowie Promotion-Workflows definiert werden.
Diese dienen als Grundlage für die Gestaltung des Prototypen
im nächsten Schritt.
\bigskip

\noindent
Soweit möglich und sinnvoll, wird für den entwickelten Prototypen
auf bestehende Toolkits und Best Practices zurückgegriffen.
Es ist erstrebenswert,
die Kubernetes-API
% TODO: cite
sowie den
GitOps-Toolkit
% TODO: cite
lediglich zu erweitern,
um so gut wie möglich nativ in das bestehende GitOps-Ökosystem 
der CNCF
integriert zu sein.
\bigskip

\noindent
Nach 
\citeauthor{HOUDE1997367} (\citeyear{HOUDE1997367})
ist ein Prototyp jede Darstellung einer Design-Idee, unabhängig vom Medium.
Ein Prototyp ist etwas, das als Modell oder Inspiration für spätere Entwicklungen dient
\autocite{HOUDE1997367}.
\bigskip

\noindent
Der entwickelte Prototyp soll im Rahmen eines Laborexperiments
den bestehenden Ansätzen zur Lösung der Problemstellung gegenübergestellt werden.
\bigskip

\noindent
\citeauthor{montgomery2017design} (\citeyear{montgomery2017design}) definiert ein Experiment als einen Test 
oder eine Reihe von Durchläufen, wobei absichtlich Änderungen an den Eingangsvariablen vorgenommen werden, 
um dann Gründe für Veränderungen am Ausgangsergebnis beobachten und identifizieren zu können 
\autocite{montgomery2017design}.

%Nach erfolgreicher Implementierung des Prototyps,
%soll mittels einer Scheduler-Extension in einem Laborexperiment gezeigt werden,
%dass Kubernetes Workloads adaptiv Nodes zugeordnet werden können,
%auf welchen diese die beste Energieeffizienz aufweisen.
%\bigskip

%Das Ziel dabei ist etwas Neues bei einem System oder Prozess zu entdecken, 
%oder eine bestehende Theorie zu beweisen
%\autocite{montgomery2017design}.
%\bigskip




