\chapter{Forschungsmethodik}

Zuerst wird bestehende wissenschaftliche Literatur zum Thema, 
sowie der konkreten Forschungsfrage herangezogen.
Weiters werden existierende Softwarelösungen,
die unter Open-Source-Lizenzen zur Verfügung stehen,
für den zweckmäßigen Einsatz für die vorliegende Forschung evaluiert.
Schließlich wird eine Baseline als State-of-the-Art definiert.
\bigskip

Sind existierende Lösungen bzw. wissenschaftliche Prototypen für das vorliegende Forschungsvorhaben geeignet,
dann werden diese genutzt und erweitert oder abgeändert um für den Zweck dieser Forschung zu dienen.
Andernfalls wird ein Prototyp für die Kubernetes-native Stromverbrauchs-Messung auf Workload-Ebene entwickelt.
\bigskip

Nach 
\citeauthor{HOUDE1997367} (\citeyear{HOUDE1997367})
ist ein Prototyp jede Darstellung einer Design-Idee, unabhängig vom Medium.
Ein Prototyp ist etwas, das als Modell oder Inspiration für spätere Entwicklungen dient
\autocite{HOUDE1997367}.
\bigskip

Nach erfolgreicher Implementierung des Prototyps,
soll mittels einer Scheduler-Extension in einem Laborexperiment gezeigt werden,
dass Kubernetes Workloads adaptiv Nodes zugeordnet werden können,
auf welchen diese die beste Energieeffizienz aufweisen.
\bigskip

%Das Ziel dabei ist etwas Neues bei einem System oder Prozess zu entdecken, 
%oder eine bestehende Theorie zu beweisen
%\autocite{montgomery2017design}.
%\bigskip

\citeauthor{montgomery2017design} (\citeyear{montgomery2017design}) definiert ein Experiment als einen Test 
oder eine Reihe von Durchläufen, wobei absichtlich Änderungen an den Eingangsvariablen vorgenommen werden, 
um dann Gründe für Veränderungen am Ausgangsergebnis beobachten und identifizieren zu können 
\autocite{montgomery2017design}.


