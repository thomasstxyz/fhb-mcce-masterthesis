\chapter{Forschungsmethodik}

In the beginning, existing literature on the topic, 
as well as the concrete research questions
are discussed.
Furthermore, various
opinions and approaches of pioneering organizations in the GitOps community are analyzed.
\bigskip

\noindent
Next, existing software solutions,
which are available under open source licenses,
will be evaluated for their appropriate use for the present research.
The aim is to find out the extent to which the defined research questions
can be addressed with existing solutions.
Finally, a baseline is defined as state-of-the-art.
\bigskip

\noindent
Furthermore, abstract models for deployment environments
and promotion workflows will be defined.
These serve as a basis for the design of the prototype in the next step.
\bigskip

\noindent
As far as possible and reasonable, existing toolkits and best practices will be used for the prototype.
It is desirable to solely extend
the Kubernetes-API
% TODO: cite
and the
GitOps-Toolkit
% TODO: cite
to integrate the prototype natively into the existing ecosystem around GitOps within the CNCF.
\bigskip

\noindent
According to 
\citeauthor{HOUDE1997367} (\citeyear{HOUDE1997367})
a prototype is any representation of a design idea, regardless of the medium.
A prototype is something that serves as a model or inspiration for later developments
\autocite{HOUDE1997367}.
\bigskip

\noindent
The developed prototype will be compared to existing approaches to solving the problem in a laboratory experiment.
\bigskip

\noindent
\citeauthor{montgomery2017design} (\citeyear{montgomery2017design})
defines an experiment as a test 
or series of runs in which intentional changes are made to the input variables, 
in order to then observe and identify reasons for changes in the output result
\autocite{montgomery2017design}.

%Nach erfolgreicher Implementierung des Prototyps,
%soll mittels einer Scheduler-Extension in einem Laborexperiment gezeigt werden,
%dass Kubernetes Workloads adaptiv Nodes zugeordnet werden können,
%auf welchen diese die beste Energieeffizienz aufweisen.
%\bigskip

%Das Ziel dabei ist etwas Neues bei einem System oder Prozess zu entdecken, 
%oder eine bestehende Theorie zu beweisen
%\autocite{montgomery2017design}.
%\bigskip




