\chapter{Discussion and Interpretation}
\label{discussion-and-interpretation}

%Results, Discussion and conclusions come here.

TODO

The overall goal of the thesis is to
provide a solution to the problem of release promotion in GitOps environments.
In this regard a prototype will be developed
and its functionality demonstrated in a proof-of-concept evaluation.
The goal for the prototype is to implement it as a modular extension to Kubernetes
and existing GitOps tooling.
By doing so, users can promote releases between multiple deployment environments
following the GitOps principles.
The main contribution of this thesis is a proof of concept for the developed prototype
which can be used as an extension to the GitOps toolkit within the CNCF.






{\color{red}TODO opinion of researcher}



\section{Learnings From The Prototype Implementation}

TODO

- ...

- User Feedback From Interview Partners

- access and security of the operator to GitOps repo








\section{Towards Standardized GitOps Promotions}

TODO






\section{Related Ideas And Approaches}

TODO

- rolling production environment

- increasing observability/overview of GitOps repositories


The conducted semi-structured interviews gave valuable insights into
the unique views of the problem of each of the working professionals.
Some of the related ideas and alternative approaches are discussed in this section.







\section{?}
































%
%\section{Instruction included in the original FHBgld word processor template}
%Die Ergebnisse der Arbeit sind in übersichtlicher Form darzustellen Die gewählte Form der Darstellung ist vom gewählten Datenmaterial und den in der Einleitung gesetzten Zielen abhängig. Die Ergebnisse sind zu interpretieren und in Bezug zum Stand des Wissens zu diskutieren. Über die Beantwortung der Forschungsfrage und die daraus gezogenen Schlussfolgerungen schließt sich der Bogen zur Einleitung. 
%
%Wichtig ist die gedanklich klare Unterscheidung zwischen der Darstellung der Ergebnisse und der Interpretation/Bewertung der Ergebnisse. 
%
%\section{Code}
%If you want to show program code within your thesis you can use the \verb|\texttt{verbatim}| environment or for a more complex display take a look at \url{https://www.overleaf.com/learn/latex/Code_listing}
%
%\begin{verbatim}
%	Text enclosed inside \texttt{verbatim} environment 
%	is printed directly 
%	and all \LaTeX{} commands are ignored.
%\end{verbatim}
%
%\begin{lstlisting}[language=Python, caption=Python example]
%	import numpy as np
%	
%	def incmatrix(genl1,genl2):
%	m = len(genl1)
%	n = len(genl2)
%	M = None #to become the incidence matrix
%	VT = np.zeros((n*m,1), int)  #dummy variable
%	
%	#compute the bitwise xor matrix
%	M1 = bitxormatrix(genl1)
%	M2 = np.triu(bitxormatrix(genl2),1) 
%	
%	for i in range(m-1):
%	for j in range(i+1, m):
%	[r,c] = np.where(M2 == M1[i,j])
%	for k in range(len(r)):
%	VT[(i)*n + r[k]] = 1;
%	VT[(i)*n + c[k]] = 1;
%	VT[(j)*n + r[k]] = 1;
%	VT[(j)*n + c[k]] = 1;
%	
%	if M is None:
%	M = np.copy(VT)
%	else:
%	M = np.concatenate((M, VT), 1)
%	
%	VT = np.zeros((n*m,1), int)
%	
%	return M
%\end{lstlisting}
