

\noindent
%From the principles of DevOps came GitOps,
%as a collection of principles and best practices
%for the operation of software systems. At the center,
%or to be precise, the source - as the single source of truth -
%stands thereby the Git Repository as the revision control system.

% First sentence of paper
Git is becoming the new home for information technology (IT) operations.
% Introducing DevOps and new practice which emerged from it
Following a core concept of DevOps
- reducing friction between engineering teams within the software development lifecycle (SDLC) -
a new practice has emerged,
which leverages the revision control system Git for IT operations.
% GitOps is ...
GitOps is a set of principles for operating and managing software systems.
% Description of GitOps
The desired state of the managed system is
- in its entirety -
defined declaratively as code,
which is continuously reconciled with the actual state by a controller.
% Description of the problem
The promotion of new software releases between several environments
shows itself to be an unresolved problem at present.
Uniform standard practices, as well as necessary tools are missing in the open source ecosystem.
% goal of the paper
This thesis aims at addressing
the promotion of releases in GitOps environments.
% what is being researched
Abstract models of deployment environments as well as promotion workflows
will be defined.
Based on these defined models,
a standardized solution for the promotion of releases
will be implemented according to the GitOps principles.
Finally, the prototype is compared to existing solutions.
% what is the purpose, how is the future better than now
% Was können wir DANN was wir JETZT noch nicht können?
The results of this research will
% simplify the process of release promotion in GitOps environments
address the given problem
by providing vendor-agnostic solutions
for modeling environments and promoting releases between them
in a "GitOps"-native approach.




%Der Leser der Kurzfassung soll verstehen, welche Problemstellung / Fragestellung durch die vorliegende Arbeit bearbeitet wird und welche Erkenntnisse und Ergebnisse vorliegen.


