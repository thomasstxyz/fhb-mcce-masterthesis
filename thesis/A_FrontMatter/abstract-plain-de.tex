% TODO:  abstract!

\noindent
Aus den Prinzipien von DevOps entstand GitOps,
als eine Sammlung von Prinzipien und Best Practices
für die Operation von Softwaresystemen. Im Zentrum
bzw. der Quelle - als einzig wahre Quelle (Single Source of Truth) -
steht dabei das Git-Repository bzw. ein Revisions-Kontrollsystem.
Der Zustand des zu verwaltenden Systems wird
vollständig deklarativ als Code definiert. Ein GitOps-Controller
kümmert sich um den kontinuierlichen Abgleich zwischen dem
definierten Ziel-Zustand und dem tatsächlichen Systemzustand.
Die Promotion neuer Software-Releases zwischen mehreren Umgebungen
zeigt sich als derzeit offenes Problem.
Es fehlen einheitliche Standardpraktiken, sowie nötige Tools in der Open-Source-Community.
Diese Arbeit hat als Ziel,
die Promotion von Releases in GitOps-Umgebungen
zu addressieren.
Es sollen abstrakte Modelle von Deployment-Umgebungen
sowie Promotion-Workflows definiert werden.
Aufbauend auf den zuvor definierten Modellen soll eine
standardisierte Lösung zur Promotion von Releases
nach den GitOps-Prinzipien implementiert,
und den bestehenden Lösungen gegenübergestellt werden.

%Der Leser der Kurzfassung soll verstehen, welche Problemstellung / Fragestellung durch die vorliegende Arbeit bearbeitet wird und welche Erkenntnisse und Ergebnisse vorliegen.


