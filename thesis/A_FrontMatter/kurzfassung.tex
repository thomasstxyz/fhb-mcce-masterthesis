\chapter{Kurzfassung}

TODO: neu

% Erster Satz des Papiers
Git wird zum neuen Zuhause für den Betrieb der Informationstechnologie (IT).
% Einführung in DevOps und die daraus entstandene neue Praxis
Einem Kernkonzept von DevOps
- der Verringerung von Reibungsverlusten zwischen Ingenieurteams innerhalb des Softwareentwicklungszyklus (SDLC) -
hat sich eine neue Praxis herausgebildet,
die das Revisionskontrollsystem Git für den IT-Betrieb nutzt.
% GitOps ist ...
GitOps ist eine Reihe von Prinzipien für den Betrieb und die Verwaltung von Softwaresystemen.
% Beschreibung von GitOps
Der gewünschte Zustand des verwalteten Systems ist
- in seiner Gesamtheit -
deklarativ als Code definiert,
der von einem Controller kontinuierlich mit dem Ist-Zustand abgeglichen wird.
% Beschreibung des Problems
Die Weitergabe von neuen Software-Releases zwischen mehreren Umgebungen
stellt sich als ein derzeit ungelöstes Problem dar.
Es fehlen im Open-Source-Ökosystem einheitliche Standardpraktiken sowie die notwendigen Werkzeuge.
% Ziel der Arbeit
Ziel dieser Arbeit war es, das Problem der
Promotion von Releases in GitOps-Umgebungen zu adressieren.
Die Problemstellung wurde identifiziert und motiviert durch
Befragung von Fachleuten aus der Praxis, die im Bereich GitOps arbeiten.
Es wurden eindeutige Problemstellungen definiert, aus denen Lösungsziele abgeleitet wurden.
% was erforscht wird
Abstrakte Modelle von Deployment-Umgebungen sowie Promotion-Workflows
wurden entworfen.
Basierend auf diesen Modellen,
wurde eine standardisierte Lösung für die Promotion von Releases
entwickelt,
in Form eines Kubernetes-Operators,
der die GitOps-Prinzipien befolgt.
Schließlich wurde die Forschung evaluiert, indem
die Funktionalität des vorgeschlagenen Prototyps
mit den Forschungszielen verglichen wurde.
% Was ist der Zweck, wie ist die Zukunft besser als jetzt
% Was können wir DANN, was wir JETZT noch nicht können?
Die Ergebnisse dieser Forschung
% Vereinfachung des Prozesses der Release-Promotion in GitOps-Umgebungen
adressieren das gegebene Problem
durch die Bereitstellung einer herstellerneutralen Lösung
für die Modellierung von Umgebungen und die Beförderung von Releases zwischen ihnen
in einem GitOps-nativen Ansatz.
Der vorgeschlagene Operator-Prototyp und seine Implementierung zusammen mit
der demonstrierten Anwendung im Proof of Concept,
beschreibt einen möglichen Weg, wie die Förderung von Releases
in GitOps-Umgebungen gestaltet werden kann.
Für zukünftige Arbeiten sollte der Prototyp verbessert werden,
indem die Benutzererfahrung und die erforderlichen Funktionen untersucht werden,
im Rahmen der designwissenschaftlichen Forschungsmethodik.





















%Hier ist der Inhalt der Arbeit in komprimierter Form darzustellen.  
%
%Länge maximal 1 Seite. 
%
%Anmerkung:  
%
%Der Leser der Kurzfassung soll verstehen, welche Problemstellung / Fragestellung durch die vorliegende Arbeit bearbeitet wird und welche Erkenntnisse und Ergebnisse vorliegen. 
%
%Bitte beachten Sie auch die Richtlinie des Studiengangs zur Erstellung von wissenschaftlichen Arbeiten, vor allem die formalen Anforderungen wie z.B. Zitierweise nach APA-Styleguide in der Form (Autor, Jahr, Seite).  
%
%Hinweis: Die Arbeit ist doppelseitig auszudrucken. Folgende Seiten beginnen IMMER auf einer rechten, ungeraden Seite: 
%
%Deckblatt 
%
%Vorwort 
%
%Inhaltsverzeichnis 
%
%Kurzfassung 
%
%Abstract 
%
%Neues Kapitel auf Kapitelebene 1. Bitte fügen Sie mit Kapitel 2 (Grundlagen) beginnend jeweils entsprechende Leerseiten ein. Alle vorhergehenden Abschnitte sind mittels eigenen Word-Abschnitten bereits festgelegt. 

%\lipsum[2-3]