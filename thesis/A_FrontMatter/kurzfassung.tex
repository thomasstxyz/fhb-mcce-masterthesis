\chapter{Kurzfassung}

In Anlehnung an ein Kernkonzept von DevOps - die Verringerung der Reibungsverluste zwischen Entwicklungsteams innerhalb des Softwareentwicklungszyklus (SDLC) - hat sich eine Deploymentpraxis herausgebildet, die das Versionskontrollsystem Git für IT-Operations nutzt. GitOps ist eine Reihe von Prinzipien für den Betrieb und die Verwaltung von Softwaresystemen. Der gewünschte Zustand des verwalteten Systems wird - in seiner Gesamtheit - deklarativ als Code definiert, der von einem Controller kontinuierlich mit dem tatsächlichen Zustand abgeglichen wird. GitOps bietet u. a. eine größere Transparenz des Infrastrukturzustands, eine sogenannte Single-Source-of-Truth mit integrierter Audit-Historie, kürzere Wiederherstellungszeiten und verbesserte Sicherheit.

Ein Problem des GitOps-Ansatzes ist die Promotion neuer Software-Releases zwischen Umgebungen aufgrund der asynchronen Charakteristik von GitOps-Deployments. Derzeit gibt es keine Standardverfahren und -werkzeuge für die Durchführung von Promotions in einer GitOps-nativen Weise. Daher neigen Anwender dazu, Workflow-/Pipeline-Systeme zu verwenden, um Promotions durchzuführen. Diese Arbeit zielt darauf ab, das Problem der Promotion von Releases in GitOps-Umgebungen zu adressieren. Die Problemstellung wurde durch die Befragung von Fachleuten, die im Bereich GitOps tätig sind, identifiziert und motiviert. Es wurden eindeutige Problemstellungen definiert, aus denen Lösungsziele abgeleitet wurden. Es wurden abstrakte Modelle von Deployment-Umgebungen sowie Promotion-Workflows entworfen. Auf der Grundlage dieser Modelle wurde eine standardisierte Lösung für die Promotion von Releases entworfen und prototypisch entwickelt, die den GitOps-Prinzipien entspricht. Die Forschung wurde evaluiert, indem die Funktionalität des vorgestellten Prototyps mit den Forschungszielen verglichen wurde.

Die Ergebnisse dieser Forschung adressieren das gegebene Problem, indem sie eine herstellerneutrale Lösung für die Modellierung von Umgebungen und die Promotion von Releases zwischen ihnen mit einem GitOps-nativen Ansatz bieten. Der vorgeschlagene Operator-Prototyp und seine Implementierung, zusammen mit der demonstrierten Verwendung im Proof of Concept, beschreibt einen möglichen Weg, wie die Promotion von Releases in GitOps-Umgebungen gestaltet werden kann. Für zukünftige Arbeiten sollte der Prototyp verbessert werden, indem die Benutzererfahrung und die gewünschten Funktionalitäten im Rahmen der Design Science Forschungsmethodik untersucht werden.




















%Hier ist der Inhalt der Arbeit in komprimierter Form darzustellen.  
%
%Länge maximal 1 Seite. 
%
%Anmerkung:  
%
%Der Leser der Kurzfassung soll verstehen, welche Problemstellung / Fragestellung durch die vorliegende Arbeit bearbeitet wird und welche Erkenntnisse und Ergebnisse vorliegen. 
%
%Bitte beachten Sie auch die Richtlinie des Studiengangs zur Erstellung von wissenschaftlichen Arbeiten, vor allem die formalen Anforderungen wie z.B. Zitierweise nach APA-Styleguide in der Form (Autor, Jahr, Seite).  
%
%Hinweis: Die Arbeit ist doppelseitig auszudrucken. Folgende Seiten beginnen IMMER auf einer rechten, ungeraden Seite: 
%
%Deckblatt 
%
%Vorwort 
%
%Inhaltsverzeichnis 
%
%Kurzfassung 
%
%Abstract 
%
%Neues Kapitel auf Kapitelebene 1. Bitte fügen Sie mit Kapitel 2 (Grundlagen) beginnend jeweils entsprechende Leerseiten ein. Alle vorhergehenden Abschnitte sind mittels eigenen Word-Abschnitten bereits festgelegt. 

%\lipsum[2-3]