	%Please consider the following section of the "Formvorschriften für die gedruckte Version"
	%Im Anhang ist eine Zusammenfassung (Abstract) mitzubinden. 
	%Ist die Arbeit in einer Fremdsprache verfasst, ist im Anhang jedenfalls eine deutsche Zusammenfassung mitzubinden.
%\chapter{Abstract}
%		This \LaTeX{} template provides example on how to format and display text, 
%		mathematical formulas, and insert tables or images. There is a lot more you 
%		can do with \LaTeX{}, for more information check out https://en.wikibooks.org/wiki/LaTeX.

%Komprimierter Inhalt der Arbeit (Kurzfassung) in Englisch. 
%
%Länge maximal 1 Seite. 

% Abstract must be written in the "past tense".

%\lipsum[2-4]





\noindent
% First sentence of paper
% ?
% Introducing DevOps and new practice which emerged from it
Following a core concept of DevOps
- reducing friction between engineering teams within the software development lifecycle (SDLC) -
a deployment practice has emerged,
which leverages the version control system Git for IT operations.
% GitOps is ...
GitOps is a set of principles for operating and managing software systems.
% Description of GitOps
The desired state of the managed system is
- in its entirety -
defined declaratively as code,
which is continuously reconciled with the actual state by a controller.
% what problem does GitOps solve?
GitOps provides greater visibility into infrastructure state,
a single source of truth with built-in audit history,
reduced time to recovery and improved security,
among other benefits.

% Description of the problem
A problem with the GitOps approach is the promotion of new software releases between environments,
due to the asynchronous nature of GitOps deployments.
Currently there is no standard practice and tooling for achieving promotions in a GitOps-native way.
Hence, users are prone to use workflow/pipeline systems to achieve promotions.
% goal of the paper
This thesis aims at addressing the problem
of promotion of releases in GitOps environments.
The problem statement was identified and motivated by
interviewing practicing professionals who work in the GitOps field.
Distinct problem items were defined, from which solution objectives were inferred.
% what is being researched
Abstract models of deployment environments as well as promotion workflows
were designed.
Based on these models,
a standardized solution for the promotion of releases
was designed and developed prototypically,
adhering to the GitOps principles.
The research was evaluated by
comparing the functionality of the proposed prototype
to the research objectives.

% what is the purpose, how is the future better than now
% Was können wir DANN was wir JETZT noch nicht können?
The results of this research
% simplify the process of release promotion in GitOps environments
address the given problem
by providing a vendor-neutral solution
for modeling environments and promoting releases between them
with a GitOps-native approach.
The proposed operator prototype and its implementation along with
the demonstrated use in the proof of concept,
describes one possible way of how the promotion of releases
in GitOps environments can be designed.
For future work, the prototype should be improved
by doing research on its user experience and desired capabilities,
within the framework of
the design science research methodology.










% setting, where are we

% aim and objectives

% therefore what was done

% the results showed

% what will be done next









