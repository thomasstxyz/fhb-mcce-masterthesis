\documentclass{beamer}

\title{\yourThesisTitle}
\subtitle{\typeOfWork}
\author{\yourNameInclTitle}

\usepackage[style=apa, backend=biber, language=ngerman]{biblatex}
\addbibresource{references.bib}
\renewcommand\bibname{\section{Literaturverzeichnis}}


\usetheme{Frankfurt}
\usepackage{xcolor}
\usepackage{textpos}
%\usepackage{enumitem}

\usepackage{graphicx}
\usepackage{subcaption}
\captionsetup{compatibility=false}

\definecolor{fhgreen}{rgb}{0,0.47,0.32}
%\definecolor{fhgreen}{rgb}{0,121,82}
\definecolor{orange}{rgb}{1,0.5,0}

\setbeamercolor{frametitle}{bg=fhgreen}
\setbeamercolor{title}{bg=fhgreen}
\setbeamercolor{itemize items}{bg=fhgreen}
\setbeamercolor{enumerate items}{bg=fhgreen}

\setbeamertemplate{itemize items}{\color{fhgreen}$\blacktriangleright$}
\setbeamertemplate{enumerate items}{\color{fhgreen}$\blacktriangleright$}

%\setbeamercolor{local structure}{fg=fhgreen}

\begin{document}

\addtobeamertemplate{frametitle}{}{%
\begin{textblock*}{100mm}(.85\textwidth,6.3cm)
%\begin{textblock*}{100mm}(.85\textwidth,-1cm)
\includegraphics[scale=0.2]{images/FH-burgenland-logo.png}
%\includegraphics[height=1cm,width=2cm]{images/FH-burgenland-logo.png}
\end{textblock*}}

% title page
\maketitle

\section{Einleitung u. Zielsetzung}

\begin{frame}
    \frametitle{Problemstellung}
    \begin{itemize}
        \item Hoher Stromverbrauch von Rechenzentren
        \item softwarebasierte Lösungen zur Verbesserung der Energieeffizienz fehlen
        \item Observability in modernen verteilten Systemen ist eine Herausforderung
    \end{itemize}
\end{frame}

\begin{frame}
    \frametitle{Zielsetzung u. Fragestellung}
    \textbf{Forschungsfrage: }
    \begin{center}
        \large{Inwieweit lässt sich der Stromverbrauch von Kubernetes Workloads bzw. Applikationen feststellen und ein effizienz-orientiertes Scheduling implementieren?}
    \end{center}
\end{frame}

%\section{Grundlagen}
%
%\begin{frame}
%    \frametitle{Grundlagen}
%    \begin{itemize}
%        \item Allgemeine Definitionen
%        \item a
%        \item b
%        \item c
%    \end{itemize}
%\end{frame}

\begin{frame}
    \frametitle{Stand des Wissens}
    \textbf{Literaturrecherche}
    \begin{itemize}
        \item \citeauthor{colmant:hal-01130030} (\citeyear{colmant:hal-01130030}) erforschen "Process-level Power Estimation in VM-based Systems"
        \item \citeauthor{fieni:hal-02470128} (\citeyear{fieni:hal-02470128}) erforschen "SmartWatts: Self-Calibrating Software-Defined Power Meter for Containers"
        \item \citeauthor{8705815} (\citeyear{8705815}) beobachten eine Reduzierung des Stromverbrauchs um 10-20 \% nach Implementierung eines intelligenten Schedulers.
    \end{itemize}
\end{frame}

\section{Methodik u. Vorgangsweise}

\begin{frame}
    \frametitle{Forschungsmethoden}
    \textbf{Design und Implementierung eines Prototypen}
    \begin{itemize}
        \item Messung des Stromverbrauchs innerhalb Kubernetes
    \end{itemize}

    \textbf{Durchführung eines Laborexperiments}
    \begin{itemize}
        \item Adaptives energieeffizienz-orientiertes Scheduling
    \end{itemize}
\end{frame}

%
%\begin{frame}
%    \frametitle{Vorgangsweise}
%    %\setlist[enumerate]{label={\arabic*.}}
%    \begin{itemize}
%        \item Bestimmung der Messgrößen
%    \end{itemize}
%    \vspace{0.2cm}
%
%    \begin{figure}[H]
%        \centering
%        \begin{subfigure}{.5\linewidth}
%            \centering
%            \includegraphics[scale=0.3]{images/input-output-factors-in-experiment.png}
%  %\caption{}
%  \label{fig:sub1}
%        \end{subfigure}%
%        \begin{subfigure}{.5\linewidth}
%            \centering
%            \includegraphics[scale=0.12]{images/120s-180s.png}
%  %\caption{}
%  \label{fig:sub2}
%        \end{subfigure}
%%\caption{byrequests vs. round-robin, 100k, 2000 Anfragen}
%\label{fig:byrequestsvs.round-robin100k2000Anfragen}
%    \end{figure}
%
%    \vspace{0.2cm}
%        100k bei 2000 Anfragen \\
%        1M bei 500 Anfragen \\
%        10M bei 50 Anfragen
%\end{frame}
%
%\begin{frame}
%    \frametitle{Vorgangsweise}
%    \begin{itemize}
%        \item Aufbau der Laborumgebung
%    \end{itemize}
%    \vspace{1cm}
%    \includegraphics[scale=0.48]{images/drawio/Virtualisierungsebenen.png}
%\end{frame}
%
%\begin{frame}
%    \frametitle{Vorgangsweise}
%    \begin{itemize}
%        \item Durchführung des Experiments
%    \end{itemize}
%    \vspace{1cm}
%    \includegraphics[scale=0.48]{images/drawio/HTTP-Anfrage.png}
%\end{frame}
%
%\begin{frame}
%    \frametitle{Vorgangsweise}
%    \begin{itemize}
%        \item Darstellung der Ergebnisse
%    \end{itemize}
%    \vspace{0.3cm}
%    \includegraphics[scale=0.19]{images/default-10M-req.png}
%\end{frame}
%
%\begin{frame}
%    \frametitle{Vorgangsweise}
%    \begin{itemize}
%        \item Bewertung und Schlussfolgerungen
%    \end{itemize}
%\end{frame}


\section{Ergebnisse u. Schlussfolgerungen}

\begin{frame}
	\frametitle{Erwartete Ergebnisse}
	\begin{itemize}
		\item Kubernetes-native Messung des Stromverbrauchs von Kubernetes Workloads
		\item Prototyp als modulare Extension zu Kubernetes
		\item Metriken über OpenTelemetry zur Verfügung stellen
		\item Evaluierung eines adaptiven effizienz-orientierten Scheduling
	\end{itemize}
\end{frame}

%\begin{frame}
%    \frametitle{Ergebnisse}
%    \includegraphics[scale=0.19]{images/default-100k-req.png}
%\end{frame}
%
%\begin{frame}
%    \frametitle{Ergebnisse}
%    \includegraphics[scale=0.19]{images/default-10M-req.png}
%\end{frame}
%
%\begin{frame}
%    \frametitle{Ergebnisse}
%    \includegraphics[scale=0.22]{images/default-100k-ram.png}
%\end{frame}
%
%\begin{frame}
%    \frametitle{Ergebnisse}
%    \includegraphics[scale=0.19]{images/1h-10M.png}
%\end{frame}

\begin{frame}
    \frametitle{Ausblick}
    \begin{itemize}
        \item Stromverbrauchs-Metriken in Kubernetes Metrics API nativ integrieren
        \item Neben CPU u. Memory auch Stromverbrauch von Pods bzw. Containern nativ in Kubernetes bereitstellen
        \item Feststellen von effizienten Umgebungen (z. B. CPU-Architektur ARM oder AMD64) ohne vorheriges Training
        \item Adaptives Scheduling von Kubernetes Workloads orientiert an Energieeffizienz
    \end{itemize}
%    \vspace{1cm}
%    - aa \\
%    - bb
\end{frame}

%        \item  -- anotheritem --

\begin{frame}
    \frametitle{Ende der Präsentation}
    Dankeschön!
\end{frame}

%% columns in beamer
%\begin{frame}
%    \frametitle{Columns}
%    \begin{columns}
%        \column{.5\textwidth}
%        1 column
%        \column{.5\textwidth}
%        2 column
%    \end{columns}
%\end{frame}

\end{document}
